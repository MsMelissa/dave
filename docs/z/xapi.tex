\documentclass[../main.tex]{subfiles}

\begin{document}

\section{xAPI Formal Specification}
The current formal specification only defines xAPI statements
abstractly within the context of Z. A concrete definition for xAPI
statements is outside the scope of this document.

\subsection{Basic and Free Types}
$[MBOX, MBOX\_SHA1SUM, OPENID, ACCOUNT]$
\begin{itemize}
\item Basic Types for the abstract representation of the different forms of Inverse Functional Identifiers found in xAPI
\end{itemize}
$[CHOICES, SCALE, SOURCE, TARGET, STEPS]$
\begin{itemize}
\item Basic Types for the abstract representation of the different forms of Interaction Components found in xAPI
\end{itemize}
$IFI$ ::=$ MBOX \,|\, MBOX\_SHA1SUM \,|\, OPENID \,|\, ACCOUNT$
\begin{itemize}
\item Free Type unique to Agents and Groups, The concrete definition of the listed Basic Types
  is outside the scope of this specification
\end{itemize}
$OBJECTTYPE$ ::=$ Agent \,|\, Group \,|\, SubStatement \,|\,
StatementRef \,|\, Activity$
\begin{itemize}
\item A type which can be present in all activities as defined by
  the xAPI specification
\end{itemize}
$INTERACTIONTYPE$ ::= $true-false \,|\, choice \,|\, fill-in \,|\,
long-fill-in \,|\, matching \,| \\ performance \,|\, sequencing \,|\,
likert \,|\, numeric \,|\, other$
\begin{itemize}
\item A type which represents the possible interactionTypes as
  defined within the xAPI specification
\end{itemize}
$INTERACTIONCOMPONENT$ ::= $CHOICES \,|\, SCALE \,|\, SOURCE \,|\,
TARGET \,|\, STEPS$
\begin{itemize}
\item A type which represents the possible interaction components as
  defined within the xAPI specification
\item the concrete definition of the listed Basic Types is outside the
  scope of this specification
\end{itemize}
$CONTEXTTYPES$ ::= $parent \,|\, grouping \,|\, category \,|\, other$
\begin{itemize}
\item A type which represents the possible context types as
  defined within the xAPI specification
\end{itemize}
$[STATEMENT]$
\begin{itemize}
\item Basic type for an xAPI data point
\end{itemize}
$[AGENT, GROUP]$
\begin{itemize}
\item Basic types for Agents and collections of Agents
\end{itemize}

\subsection{Id Schema}
\begin{schema}{Id}
  id : \finset_1 \#1
\end{schema}
\begin{itemize}
\item the schema $Id$ introduces the component $id$ which is a
  non-empty, finite set of 1 value
\end{itemize}

\subsection{Schemas for Agents, Groups and Actors}

\begin{schema}{Agent}
  agent : AGENT \\
  objectType : OBJECTTYPE \\
  name : \finset_1 \#1 \\
  ifi : IFI
  \where
  objectType = Agent \\
  agent = \{ifi\} \cup \power \{name, objectType\}
\end{schema}
\begin{itemize}
\item The schema $Agent$ introduces the component $agent$ which is a set
  consisting of an $ifi$ and optionally an $objectType$ and/or $name$
\end{itemize}

\begin{schema}{Member}
  Agent \\
  member : \finset_1
  \where
  member = \{a : AGENT \,|\, \forall a_{n}: a_{i}..a_{j} @ i \leq n
  \leq j @ a = agent\}
\end{schema}
\begin{itemize}
\item The schema $Member$ introduces the component $member$ which is a set of
  objects $a$, where for every $a$ within $a_{0}..a_{n}$, $a$ is an $agent$
\end{itemize}

\begin{schema}{Group}
  Member \\
  group : GROUP \\
  objectType : OBJECTTYPE \\
  ifi : IFI\\
  name : \finset_1 \#1
  \where
  objectType = Group \\
  group = \{objectType, name, member\} \lor \{objectType, member\}
  \lor \\ \t2 \{objectType, ifi\} \cup \power \{name, member\}
\end{schema}
\begin{itemize}
\item The schema $Group$ introduces the component $group$ which is of
  type $GROUP$ and is a set of either $objectType$ and $member$ with optionaly $name$ or
  $objectType$ and $ifi$ with optionally $name$ and/or $member$
\end{itemize}

\begin{schema}{Actor}
  Agent \\
  Group \\
  actor : AGENT \lor GROUP
  \where
  actor = agent \lor group
\end{schema}
\begin{itemize}
\item The schema $Actor$ introduces the component $actor$ which
  is either an $agent$ or $group$
\end{itemize}


\subsection{Verb Schema}
\begin{schema}{Verb}
  Id \\
  display, verb : \finset_1
  \where
  verb = \{id, display\} \lor \{id\}
\end{schema}
\begin{itemize}
\item The schema $Verb$ introduces the component $verb$ which
  is a set that consists of either $id$ and the non-empty, finite set
  $display$ or just $id$
\end{itemize}

\subsection{Object Schema}

\begin{schema}{Extensions}
  extensions, extensionVal : \finset_1 \\
  extensionId : \finset_1 \#1 \\
  \where
  extensions = \{e : (extensionId, extensionVal)\ \,|\,
  \forall e_{n} : e_{i}..e_{j} @ i \leq n \leq j @ \\
  \t3 \, (extensionId_{i}, extensionVal_{i})
  \lor (extensionId_{i}, extensionVal_{j}) \land \\
  \t3 \, (extensionId_{j}, extensionVal_{i})
  \lor (extensionId_{j}, extensionVal_{j})
  \land \\ \t3 \, extensionId_{i} \not = extensionId_{j}\}
\end{schema}
\begin{itemize}
\item The schema $Extensions$ introduces the component $extensions$ which
  is a non-empty, finite set that consists of ordered pairs of
  $extensionId$ and $extensionVal$. Different $extensionId$s can
  have the same $extensionVal$ but there can not be two identical
  $extensionId$ values
\item $extensionId$ is a non-empty, finite set with one value
\item $extensionVal$ is a non-empty, finite set
\end{itemize}

\begin{schema}{InteractionActivity}
  interactionType : INTERACTIONTYPE \\
  correctResponsePattern : \seq_1 \\
  interactionComponent: INTERACTIONCOMPONENT \\
  \where
  interactionActivity = \{interactionType, correctReponsePattern,
  interactionComponent\} \lor \\ \t5 \{interactionType, correctResponsePattern\}
\end{schema}
\begin{itemize}
\item The schema $InteractionActivity$ introduces the component
  $interactionActivity$ which is a set of either $interactionType$
  and $correctResponsePattern$ or $interactionType$ and
  $correctResponsePattern$ and $interactionComponent$
\end{itemize}

\begin{schema}{Definition}
  InteractionActivity \\
  Extensions \\
  definition, name, description : \finset_1 \\
  type, moreInfo : \finset_1 \#1
  \where
  definition = \power_1 \{name, description, type, moreInfo,
  extensions, interactionActivity\} \\
\end{schema}
\begin{itemize}
\item The schema $Definition$ introduces the component
  $definition$ which is the non-empty, finite power set of $name$, $description$,
  $type$, $moreInfo$ and $extensions$
\end{itemize}

\begin{schema}{Object}
  Id \\
  Definition \\
  Agent \\
  Group \\
  Statement \\
  objectTypeA, objectTypeS, objectTypeSub, objectType  : OBJECTTYPE \\
  substatement : STATEMENT \\
  object : \finset_1 \\
  \where
  substatement = statement \\
  objectTypeA = Activity \\
  objectTypeS = StatementRef \\
  objectTypeSub = SubStatement \\
  objectType = objectTypeA \lor objectTypeS \\
  object = \{id\} \lor \{id, objectType\} \lor \{id, objectTypeA,
  definition\} \\ \t2 \lor \{id, definition\} \lor \{agent\} \lor
  \{group\} \lor \{objectTypeSub, substatement\} \\
  \t2 \lor \{id, objectTypeA\}
\end{schema}
\begin{itemize}
\item The schema $Object$ introduces the component $object$ which
  is a non-empty, finite set of either $id$, $id$ and $objectType$,
  $id$ and $objectTypeA$, $id$ and $objectTypeA$ and $definition$,
  $agent$, $group$, or $substatement$
\item The schema $Statement$ and the corresponding component
  $statement$ will be defined later on in this specification
\end{itemize}

$\\\\\\\\\\\\$ %%% Header with content

\subsection{Result Schema}

\begin{schema}{Score}
  score : \finset_1 \\
  scaled, min, max, raw : \num \\
  \where
  scaled = \{ n : \num \,|\, -1.0 \leq n \leq 1.0 \} \\
  min = n < max \\
  max = n > min \\
  raw = \{ n : \num \,|\, min \leq n \leq max \} \\
  score = \power_1 \{scaled, raw, min, max\}
\end{schema}
\begin{itemize}
\item The schema $Score$ introduces the component $score$ which is
  the non-empty powerset of $min$, $max$, $raw$ and $scaled$
\end{itemize}

\begin{schema}{Result}
  Score \\
  Extensions \\
  success, completion, response, duration : \finset_1 \#1 \\
  result : \finset_1
  \where
  success = \{true\} \lor \{false\} \\
  completion = \{true\} \lor \{false\} \\
  result = \power_1 \{score, success, completion, response,
  duration, extensions\}
\end{schema}
\begin{itemize}
\item The schema $Result$ introduces the component $result$ which is
  the non-empty power set of $score$, $success$, $completion$,
  $response$, $duration$ and $extensions$
\end{itemize}

\subsection{Context Schema}

\begin{schema}{Instructor}
  Agent \\
  Group \\
  instructor : AGENT \lor GROUP
  \where
  instructor = agent \lor group
\end{schema}
\begin{itemize}
\item The schema $Instructor$ introduces the component $instructor$
  which can be either an $agent$ or a $group$
\end{itemize}

\begin{schema}{Team}
  Group \\
  team : GROUP
  \where
  team = group
\end{schema}
\begin{itemize}
\item The schema $Team$ introduces the component $team$ which is a $group$
\end{itemize}

\begin{schema}{Context}
  Instructor \\
  Team \\
  Object \\
  Extensions \\
  registration, revision, platform, language : \finset_1 \#1 \\
  parentT, groupingT, categoryT, otherT : CONTEXTTYPES \\
  contextActivities, statement : \finset_1
  \where
  statement = object \hide (id, objectType, agent, group,
  definition) \\
  parentT = parent \\
  groupingT = grouping \\
  categoryT = category \\
  otherT = other \\
  contextActivity = \{ca : object \hide (agent, group, objectType,
  objectTypeSub, substatement)\} \\
  contextActivityParent = (parentT, contextActivity) \\
  contextActivityCategory = (categoryT, contextActivity) \\
  contextActivityGrouping = (groupingT, contextActivity) \\
  contextActivityOther = (otherT, contextActivity) \\
  contextActivities = \power_1 \{contextActivityParent,
  contextActivityCategory, \\ \t5 \:\: contextActivityGrouping,
  contextActivityOther\} \\
  context = \power_1 \{registration, instructor, team,
  contextActivities, revision, \\ \t3 platform, language, statement, extensions\}
\end{schema}
\begin{itemize}
\item The schema $Context$ introduces the component $context$
  which is the non-empty powerset of $registration$, $instructor$,
  $team$, $contextActivities$, $revision$, $platform$, $language$,
  $statement$ and $extensions$
\item The notation $object \hide agent$ represents the component
  $object$ except for its subcomponent $agent$
\end{itemize}

\subsection{Timestamp and Stored Schema}

\begin{schema}{Timestamp}
  timestamp : \finset_1 \#1
\end{schema}

\begin{schema}{Stored}
  stored : \finset_1 \#1
\end{schema}
\begin{itemize}
\item The schema $Timestamp$ and $stored$ introduce the components
  $timestamp$ and $stored$ respectively. Each are non-empty, finite
  sets containing one value
\end{itemize}

\subsection{Attachments Schema}

\begin{schema}{Attachments}
  display, description, attachment, attachments: \finset_1 \\
  usageType, sha2, fileUrl, contentType : \finset_1 \#1 \\
  length : \nat
  \where
  attachment = \{usageType, display, contentType, length, sha2 \}
  \cup \power \{description, fileUrl\} \\
  attachments = \{a : attachment\}
\end{schema}
\begin{itemize}
\item The schema $Attachments$ introduces the component
  $attachments$ which is a non-empty, finite set of the component
  $attachment$
\item The component $attachment$ is a non-empty, finite set of the
  components $usageType$, $display$, $contentType$, $length$,
  $sha2$ with optionally $description$ and/or $fileUrl$
\end{itemize}

\subsection{Statement and Statements Schema}

\begin{schema}{Statement}
  Id \\
  Actor \\
  Verb \\
  Object \\
  Result \\
  Context \\
  Timestamp \\
  Stored \\
  Attachments \\
  statement : STATEMENT
  \where
  statement = \{actor, verb, object, stored\} \cup \\\t3 \power \{\id,
  result, context, timestamp, attachments \} \\
\end{schema}
\begin{itemize}
\item The schema $Statement$ introduces the component $statement$
  which consists of the components $actor$, $verb$, $object$ and
  $stored$ and the optional components $id$, $result$, $context$,
  $timestamp$, and/or $attachments$
\item The schema $Statement$ allows for subcomponent of $statement$
  to referenced via the $.$ (selection) operator
\end{itemize}

\begin{schema}{Statements}
  Statement \\
  IsoToUnix \\
  statements : \finset_1
  \where
  statements = \{s : statement \,|\, \forall s_{n}: s_{i}..s_{j} @ i
  \leq n \leq j \\\t3\: @ convert(s_{i}.timestamp) \leq convert(s_{j}.timestamp) \}
\end{schema}
\begin{itemize}
\item The schema $Statements$ introduces the component $statements$
  which is a non-empty, finite set of the component $statement$ which
  are in chronological order.
\end{itemize}

\end{document}
