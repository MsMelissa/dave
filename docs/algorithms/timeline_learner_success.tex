\documentclass{article}
%%% Package imports
\usepackage[ruled]{algorithm2e}
\usepackage{hyperref}
\usepackage{amsmath}
\usepackage{zed-csp}
\usepackage{breqn}
\usepackage{xcolor}
\usepackage{listings}
\lstset{literate = {-}{-}1} % get dashs to show up

%%% https://tex.stackexchange.com/questions/195486/how-can-i-highlight-json-string-values-but-not-attributes/195540#195540
%%% ^ JSON style for lstset
\newcommand\JSONnumbervaluestyle{\color{blue}}
\newcommand\JSONstringvaluestyle{\color{red}}
\newif\ifcolonfoundonthisline
\makeatletter

\lstdefinestyle{json}
{
  showstringspaces    = false,
  keywords            = {false,true},
  alsoletter          = 0123456789.,
  morestring          = [s]{"}{"},
  stringstyle         = \ifcolonfoundonthisline\JSONstringvaluestyle\fi,
  MoreSelectCharTable =%
    \lst@DefSaveDef{`:}\colon@json{\processColon@json},
  basicstyle          = \ttfamily,
  keywordstyle        = \ttfamily\bfseries,
}

\newcommand\processColon@json{%
  \colon@json%
  \ifnum\lst@mode=\lst@Pmode%
    \global\colonfoundonthislinetrue%
  \fi
}

\lst@AddToHook{Output}{%
  \ifcolonfoundonthisline%
    \ifnum\lst@mode=\lst@Pmode%
      \def\lst@thestyle{\JSONnumbervaluestyle}%
    \fi
  \fi
  %override by keyword style if a keyword is detected!
  \lsthk@DetectKeywords%
}

% reset the switch at the end of line
\lst@AddToHook{EOL}%
  {\global\colonfoundonthislinefalse}

\makeatother

\begin{document}
%%% TODO: update this document to be organized
%%% 1) xAPI Data Retrival
%%% 2) xAPI Z Specifications
%%% 3) Algorithm Name with all associated sub sections
%%% 4) Next Algorithm name with all associated sub sections
%%% ...
\section*{Timeline Of Learner Success}
As learners engage in a blended eLearning ecosystem, they will build up a history of learning experiences. When that eLearning ecosystem adheres to a framework dedicated to supporting and understanding the learner, such as the Total Learning Architecture (TLA), it becomes possible to retell their story through data. One important aspect of that story is the learner's history of success.

\section{Ideal Statements}
In order to accurately portray a learner's timeline of success, there are a few requirements of the data produced by a Learning Record Provider (LRP). They are as follows:
\begin{itemize}
\item the learner must be uniquely and consistently identified across all LRPs
\item learning activities which access a learner's understanding of material should report if the learner was successful or not
  \begin{itemize}
  \item if the assessment is scored, the grade earned by the learner should be reported
  \item if the assessment is scored, the minimum and maximum possible grade should be reported
  \end{itemize}
\item The learning activities must be uniquely and consistently identified across all LRPs
\item The time at which a learner completed a learning activity must be recorded
  \begin{itemize}
  \item The timestamp should contain an appropriate level of specificity.
  \item ie. Year, Month, Day, Hour, Minute, Second, Timezone
  \end{itemize}
\end{itemize}

\subsection{statement parameters to utilize}
The statement parameter locations here are written in
\href{http://goessner.net/articles/JsonPath/}{JSONPath}. This notation
is also compatable with the xAPI Z notation
\begin{itemize}
  \item $\$.timestamp$
  \item $\$.result.success$
  \item $\$.actor$
  \item $\$.verb.id$
  \end{itemize}

\section{TLA Statement problems}
The data collected at the TLA pilot run supports the following
algorithm.

\section{Algorithm}

\subsection{Summary}
\begin{enumerate}
  \item Query an LRS via a \href{https://github.com/adlnet/xAPI-Spec/blob/master/xAPI-Communication.md#213-get-statements}{GET} request to the statements endpoint using the parameters agent, since and until
  \item Filter the results to the set of statements where:
    \begin{itemize}
    \item $\$.verb.id$ is one of:
      \begin{itemize}
      \item http://adlnet.gov/expapi/verbs/passed
      \item https://w3id.org/xapi/dod-isd/verbs/answered
      \item http://adlnet.gov/expapi/verbs/completed
      \end{itemize}
    \item $\$.result.success$ is true
    \end{itemize}
  \end{enumerate}

  \subsection{Query an LRS via REST}
  How to query an LRS via a GET request to the Statements Resource
  \footnote{\label{moreLink} $S$ is the set of all statements parsed
    from the statements array within the HTTP response to the Curl
    request. It may be possible that multiple Curl requests are needed
    to retrieve all query results. If multiple requests are necessary,
    $S$ is the result of concatenating the result of each request into
    a single set}
  \footnote{\label{noZ} Querying an LRS will not be defined within the
  following Z specifications but the Results of the Query will be}
  %%TODO: use markdown instead
  \begin{lstlisting}[frame=single]
Agent = "agent={"account":
                {"homePage": "https://example.homepage",
                 "name": 123456}}"

Since = "since=2018-07-20T12:08:47Z"

Until = "until=2018-07-21T12:08:47Z"

Base = "https://example.endpoint/statements?"

endpoint = Base + Agent + "&" + Since + "&" + Until

Auth = Hash generated from basic auth

S = curl -X GET -H "Authorization: Auth"
         -H "Content-Type: application/json"
         -H "X-Experience-API-Version: 1.0.3"
         Endpoint
  \end{lstlisting}

  \subsection{xAPI Z Specifications}

  An xAPI statement(s) is only defined abstractly within the context
  of Z. A concrete definition for an xAPI statement(s) it outside the
  scope of this specification.

  \subsubsection{Basic Types}
  $IFI$ ::=$ mbox \,|\, mbox\_sha1sum \,|\, openid \,|\, account$
  \begin{itemize}
  \item Type unique to Agents and Groups, The concrete definition of the listed values
    is outside the scope of this specification
  \end{itemize}
  $OBJECTTYPE$ ::=$ Agent \,|\, Group \,|\, SubStatement \,|\,
  StatementRef \,|\, Activity$
  \begin{itemize}
  \item A type which can be present in all activities as defined by
    the xAPI specification
  \end{itemize}
  $INTERACTIONTYPE$ ::= $true-false \,|\, choice \,|\, fill-in \,|\,
  long-fill-in \,|\, matching \,| \\ performance \,|\, sequencing \,|\,
  likert \,|\, numeric \,|\, other$
  \begin{itemize}
  \item A type which represents the possible interactionTypes as
      defined within the xAPI specification
  \end{itemize}
  $INTERACTIONCOMPONENT$ ::= $choices \,|\, scale \,|\, source \,|\,
  target \,|\, steps$
  \begin{itemize}
  \item A type which represents the possible interaction components as
    defined within the xAPI specification
  \item the concrete definition of the listed values is outside the
    scope of this specification
  \end{itemize}
  $CONTEXTTYPES$ ::= $parent \,|\, grouping \,|\, category \,|\, other$
  \begin{itemize}
  \item A type which represents the possible context types as
    defined within the xAPI specification
  \end{itemize}
  $[STATEMENT]$
  \begin{itemize}
  \item Basic types for the results of querying an LRS
  \end{itemize}
  $[AGENT, GROUP]$
  \begin{itemize}
  \item Basic types for Agents and collections of Agents
  \end{itemize}

  \subsubsection{Id Schema}
  \begin{schema}{Id}
    id : \finset_1 \#1
  \end{schema}
  \begin{itemize}
  \item the schema $Id$ introduces the component $id$ which is a
      non-empty finite set of 1 value
  \end{itemize}

  \subsubsection{Schemas for Agents and Groups}

  \begin{schema}{Agent}
    agent : AGENT \\
    objectType : OBJECTTYPE \\
    name : \finset_1 \#1 \\
    ifi : IFI
    \where
    objectType = Agent \\
    agent = \{ifi\} \cup \power \{name, objectType\}
  \end{schema}
  \begin{itemize}
  \item The schema $Agent$ introduces the component $agent$ which is a set
    consisting of an $ifi$ and optionally an $objectType$ and/or $name$
  \end{itemize}

  \begin{schema}{Member}
    Agent \\
    member : \finset_1
    \where
    member = \{a : AGENT \,|\, \forall a: a_{0}..a_{n} @ a = agent\}
  \end{schema}
  \begin{itemize}
  \item The schema $Member$ introduces the component $member$ which is a set of
    objects $a$, where for every $a$ within $a_{0}..a_{n}$, $a$ is an $agent$
  \end{itemize}

  \begin{schema}{Group}
    Member \\
    group : GROUP \\
    objectType : OBJECTTYPE \\
    ifi : IFI
    \where
    name : \finset_1 \#1\\
    objectType = Group \\
    group = \{objectType, name, member\} \lor \{objectType, member\}
    \lor \\ \t2 \{objectType, ifi\} \cup \power \{name, member\}
  \end{schema}
  \begin{itemize}
  \item The schema $Group$ introduces the component $group$ which is of
    type $GROUP$ and is a set of either $objectType$ and $member$ with optionaly $name$ or
    $objectType$ and $ifi$ with optionally $name$ and/or $member$
  \end{itemize}

  \begin{schema}{Actor}
    Agent \\
    Group \\
    actor : AGENT \lor GROUP
    \where
    actor = agent \lor group
  \end{schema}
  \begin{itemize}
  \item The schema $Actor$ introduces the component $actor$ which
    is either an $agent$ or $group$
  \end{itemize}

  \subsubsection{Verb Schema}
  \begin{schema}{Verb}
    Id \\
    display, verb : \finset_1
    \where
    verb = \{id, display\} \lor \{id\}
  \end{schema}
  \begin{itemize}
  \item The schema $Verb$ introduces the component $verb$ which
      is a set that consists of either $id$ and the finite set
      $display$ or just $id$
  \end{itemize}

  \subsubsection{Object Schema}

  \begin{schema}{Extensions}
    extensions, extensionVal : \finset_1 \\
    extensionId : \finset_1 \#1 \\
    \where
    extensions = \{e : (extensionId, extensionVal)\ \,|\,
    \forall i,j : e_{i}..e_{j} @ \\
    \t3 \, (extensionId_{i}, extensionVal_{i})
    \lor (extensionId_{i}, extensionVal_{j}) \land \\
    \t3 \, (extensionId_{j}, extensionVal_{i})
    \lor (extensionId_{j}, extensionVal_{j})
    \land \\ \t3 \, extensionId_{i} \not = extensionId_{j}\}
  \end{schema}
  \begin{itemize}
  \item The schema $Extensions$ introduces the component $extensions$ which
      is a non-empty finite set that consists of ordered pairs of
      $extensionId$ and $extensionVal$. Different $extensionId$s can
      have the same $extensionVal$ but there can not be two identical
      $extensionId$ values
  \item $extensionId$ is a non-empty finite set with one value
  \item $extensionVal$ is a non-empty finite set
  \end{itemize}

  \begin{schema}{InteractionActivity}
    interactionType : INTERACTIONTYPE \\
    correctResponsePattern : \seq_1 \\
    interactionComponent: INTERACTIONCOMPONENT \\
    \where
    interactionActivity = \{interactionType, correctReponsePattern,
    interactionComponent\} \lor \\ \t5 \{interactionType, correctResponsePattern\}
  \end{schema}
  \begin{itemize}
  \item The schema $InteractionActivity$ introduces the component
    $interactionActivity$ which is a set of either $interactionType$
    and $correctResponsePattern$ or $interactionType$ and
    $correctResponsePattern$ and $interactionComponent$
  \end{itemize}

  \begin{schema}{Definition}
    InteractionActivity \\
    Extensions \\
    definition, name, description : \finset_1 \\
    type, moreInfo : \finset_1 \#1
    \where
    definition = \power_1 \{name, description, type, moreInfo,
    extensions, interactionActivity\} \\
  \end{schema}
  \begin{itemize}
    \item The schema $Definition$ introduces the component
      $definition$ which is the non-empty, finite power set of $name$, $description$,
      $type$, $moreInfo$ and $extensions$
  \end{itemize}

  \begin{schema}{Object}
    Id \\
    Definition \\
    Agent \\
    Group \\
    Statement \\
    objectTypeA, objectTypeS, objectTypeSub, objectType  : OBJECTTYPE \\
    substatement : STATEMENT \\
    object : \finset_1 \\
    \where
    substatement = statement \\
    objectTypeA = Activity \\
    objectTypeS = StatementRef \\
    objectTypeSub = SubStatement \\
    objectType = objectTypeA \lor objectTypeS \\
    object = \{id\} \lor \{id, objectType\} \lor \{id, objectTypeA,
    definition\} \\ \t2 \lor \{id, definition\} \lor \{agent\} \lor
    \{group\} \lor \{objectTypeSub, substatement\} \\
    \t2 \lor \{id, objectTypeA\}
  \end{schema}
  \begin{itemize}
    \item The schema $Object$ introduces the component $object$ which
      is a non-empty finite set of either $id$, $id$ and $objectType$,
      $id$ and $objectTypeA$ and $definition$, $agent$, $group$, or
      $substatement$
    \item The schema $Statement$ and the corresponding component
      $statement$ will be defined later on in this specification
  \end{itemize}

  \subsubsection{Result Schema}

  \begin{schema}{Score}
    score : \finset_1 \\
    scaled, min, max, raw : \num \\
    \where
    scaled = \{ n : \num \,|\, -1.0 \leq n \leq 1.0 \} \\
    min = n < max \\
    max = n > min \\
    raw = \{ n : \num \,|\, min \leq n \leq max \} \\
    score = \power_1 \{scaled, raw, min, max\}
  \end{schema}
  \begin{itemize}
  \item The schema $Score$ introduces the component $score$ which is
    the non-empty powerset of $min$, $max$, $raw$ and $scaled$
  \end{itemize}

  \begin{schema}{Result}
    Score \\
    Extensions \\
    success, completion, response, duration : \finset_1 \#1 \\
    result : \finset_1
    \where
    success = \{true\} \lor \{false\} \\
    completion = \{true\} \lor \{false\} \\
    result = \power_1 \{score, success, completion, response,
    duration, extensions\}
  \end{schema}
  \begin{itemize}
  \item The schema $Result$ introduces the component $result$ which is
    the non-empty power set of $score$, $success$, $completion$,
    $response$, $duration$ and $extensions$
  \end{itemize}

  \subsubsection{Context Schema}

  \begin{schema}{Instructor}
    Agent \\
    Group \\
    instructor : AGENT \lor GROUP
    \where
    instructor = agent \lor group
  \end{schema}
  \begin{itemize}
  \item The schema $Instructor$ introduces the component $instructor$
    which can be ether an $agent$ or a $group$
  \end{itemize}

  \begin{schema}{Team}
    Group \\
    team : GROUP
    \where
    team = group
  \end{schema}
  \begin{itemize}
  \item The schema $Team$ introduces the component $team$ which is a $group$
  \end{itemize}

  \begin{schema}{Context}
    Instructor \\
    Team \\
    Object \\
    Extensions \\
    registration, revision, platform, language : \finset_1 \#1 \\
    parentT, groupingT, categoryT, otherT : CONTEXTTYPES \\
    contextActivities, statement : \finset_1
    \where
    statement = object \hide (id, objectType, agent, group,
    definition) \\
    parentT = parent \\
    groupingT = grouping \\
    categoryT = category \\
    otherT = other \\
    contextActivity = \{ca : object \hide (agent, group, objectType,
    objectTypeSub, substatement)\} \\
    contextActivityParent = (parentT, contextActivity) \\
    contextActivityCategory = (categoryT, contextActivity) \\
    contextActivityGrouping = (groupingT, contextActivity) \\
    contextActivityOther = (otherT, contextActivity) \\
    contextActivities = \power_1 \{contextActivityParent,
    contextActivityCategory, \\ \t5 \:\: contextActivityGrouping,
    contextActivityOther\} \\
    context = \power_1 \{registration, instructor, team,
    contextActivities, revision, \\ \t3 platform, language, statement, extensions\}
  \end{schema}
  \begin{itemize}
  \item The schema $Context$ introduces the component $context$
    which is the non-empty powerset of $registration$, $instructor$,
    $team$, $contextActivities$, $revision$, $platform$, $language$,
    $statement$ and $extensions$
  \end{itemize}

  \subsubsection{Timestamp and Stored Schema}

  \begin{schema}{Timestamp}
    timestamp : \finset_1 \#1
  \end{schema}

  \begin{schema}{Stored}
    stored : \finset_1 \#1
  \end{schema}
  \begin{itemize}
  \item The schema $Timestamp$ and $stored$ introduce the components
    $timestamp$ and $stored$ respectively. Each are non-empty finite
    sets containing one value
  \end{itemize}

  \subsubsection{Attachements Schema}

  \begin{schema}{Attachments}
    display, description, attachment, attachments: \finset_1 \\
    usageType, sha2, fileUrl, contexntType : \finset_1 \#1 \\
    length : \nat
    \where
    attachment = \{usageType, display, contentType, length, sha2 \}
    \cup \power \{description, fileUrl\} \\
    attachments = \{a : attachment\}
  \end{schema}

  \subsubsection{Statement and Statements Schema}

  \begin{schema}{Statement}
    Id \\
    Actor \\
    Verb \\
    Object \\
    Result \\
    Context \\
    Timestamp \\
    Stored \\
    Attachements \\
    statement,\$ : STATEMENT

    \where
    statement = \{actor, verb, object, stored\} \cup \\\t3 \power \{\id,
    result, context, timestamp, attachments \} \\
  \end{schema}
  \begin{itemize}
  \item The schema $Statement$ introduces the component $statement$
    which consists of the components $actor$, $verb$, $object$ and
    $stored$ and the optional components $id$, $result$, $context$,
    $timestamp$, and/or $attachments$
  \item The schema $Statement$ allows for subcomponent of $statement$
    to refrenced via the $.$ (selection) operator
  \end{itemize}

  \begin{schema}{Statements}
    Statement \\
    statements : \finset
    \where
    statements = \{s : statement \}
  \end{schema}
  \begin{itemize}
  \item The schema $Statements$ introduces the component $statements$
    which is a non-empty finite set of components $statement$
 \end{itemize}

 \subsection{Timeline Learner Success Z Specifications}
 The following Z Schemas define the system state, initialization and
 operations necessary to perform the Timeline Learner Success Algorithm

 \subsubsection{Timeline Leaner Success Basic Types}

 $COMPLETION$ :== \\ $\{http://adlnet.gov/expapi/verbs/passed\} \, | \\
 \{https://w3id.org/xapi/dod-isd/verbs/answered\} \; | \\
 \{http://adlnet.gov/expapi/verbs/completed$\} \\
 \\
 $SUCCESS$ :== $\{true\}$

 \subsubsection{Timeline Leaner Success System State}

 \begin{schema}{TimelineLearnerSuccess}
   Statements \\
   S_{all} : \finset_1 \\
   S_{completion},S_{success} : \finset \\
   \where
   S_{all} = statements \\
   S_{completion} \subseteq S_{all} \\
   S_{success} \subseteq S_{completion}
 \end{schema}
  \begin{itemize}
  \item The set $S_{all}$ is a non-empty finite set and is the
    component $statements$
  \item The sets $S_{completion}$ and $S_{success}$ are both finite sets
  \item the set $S_{completion}$ is a subset of $S_{all}$
  \item the set $S_{success}$ is a subset of $S_{completion}$
  \end{itemize}

  \subsubsection{Initial State of Timeline Learner Success System}
  \begin{schema}{InitTimelineLearnerSuccess}
    TimelineLearnerSuccess \\
    \where
    S_{all} \not = \emptyset \\
    S_{completion} = \emptyset \\
    S_{success} = \emptyset \\
  \end{schema}
  \begin{itemize}
  \item The set $S_{all}$ is a non-empty set
  \item The sets $S_{completion}$,\,$S_{success}$ are both initially empty
  \end{itemize}

  \subsubsection{Filter for Completion}
  \begin{schema}{Completion}
    Statement \\
    completion : STATEMENT \pfun \finset \\
    s? : STATEMENT \\
    s! : \finset \\
    \where
    s? = statement \\
    s! = completion(s?) \\
    completion(s?) = \IF s?.verb.id : COMPLETION \THEN s! = s? \ELSE
    s! = \emptyset
  \end{schema}
  \begin{itemize}
  \item The schema $Completion$ inroduces the function $completion$
    which takes in the variable $s?$ and returns the variable $s!$
  \item The variable $s?$ is the component $statement$
  \item $s!$ is equal to $s?$
    if $\$.verb.id$ is of the type $COMPLETION$ otherwise $s!$ is an empty set
  \end{itemize}

  \begin{schema}{FilterForCompletion}
    \Delta TimelineLearnerSuccess \\
    Completion \\
    completions : \finset
    \where
    completions = S_{all} \\
    completions' = \{s : STATEMENT \,|\, completion(s) \not = \emptyset\} \\
    S_{completion}' = S_{completion} \cup completions' \\
  \end{schema}
  \begin{itemize}
  \item the set $completions$ is the set $S_{all}$
  \item The set $completions'$ is the set of all statements $s$ where
    the result of $completion(s)$ is not an empty set
  \item the updated set $S'_{completion}$ is the union of the previous
    state of set $S_{completion}$ and the set $completions$
  \end{itemize}

  \subsubsection{Filter for Success}
  \begin{schema}{Success}
    Statement \\
    success : STATEMENT \pfun \finset \\
    s? : STATEMENT \\
    s! : \finset \\
    \where
    s? = statement \\
    s! = success(s?) \\
    success(s?) = \IF s?.result.success : SUCCESS \THEN s! = s? \ELSE
    s! = \emptyset

  \end{schema}
  \begin{itemize}
    \item the schema $Success$ introduces the function $success$ which
      takes in the variable $s?$ and returns the variable $s!$
    \item the variable $s?$ is the component $statement$
    \item $s!$ is equal to $s?$ if $\$.result.success$ is of the type
      $SUCCESS$ otherwise $s!$ is an empty set
  \end{itemize}

  \begin{schema}{FilterForSuccess}
    \Delta TimelineLearnerSuccess \\
    Success \\
    successes : \finset
    \where
    successes = S_{completion} \\
    successes' = \{s : STATEMENT \,|\, success(s) \not = \emptyset\} \\
    S_{success}' = S_{success} \cup successes
  \end{schema}
  \begin{itemize}
  \item the set $successes$ is the set $S_{completion}$
  \item The set $successes'$ contains elements $s$ of type $STATEMENT$
    where $success(s)$ is not an empty set
  \item The updated set $S_{success}'$ is the union of the previous
    state of $S_{success}$ and $successes$
  \end{itemize}

  \subsubsection{Sequence of Operations}
  $FilterStatements \defs FilterForCompletion \semi FilterForSuccess$
  \begin{itemize}
    \item The schema $FilterStatements$ is the sequential composition
      of operation schemas $FilterForCompletion$ and
      $FilterForSuccess$
    \item $FilterForCompletion$ happens before $FilterForSuccess$
  \end{itemize}

  \subsubsection{Return}
  \begin{schema}{Return}
    \Xi TimelineLearnerSuccess \\
    FilterStatements \\
    S_{success}! : \finset
    \where
    S_{success}! = S_{success}
  \end{schema}
  \begin{itemize}
    \item The returned variable $S_{success}!$ is equal to the current
      state of variable $S_{success}$ after the operations
      $FilterForCompletion$ and $FilterForSuccess$
    \end{itemize}

  \subsection{Pseudocode}

  \begin{algorithm}[H]
    \SetAlgoLined
    \KwIn{$S_{all}$}
    \KwResult{$S_{success}$}
    \While{$S_{all}$ is not empty} {
      for each statement $s$ in $S_{all}$ \\
      \eIf {$s.verb.id$ = $COMPLETION$}
      {
        add $s$ to $S_{completion}$
      }
      {
        noop
      }}
    \While {$S_{completion}$ is not empty} {
      for each statement $sc$ in $S_{completion}$ \\
      \eIf {$sc.result.success$ = $SUCCESS$}
      {
        add $sc$ to $S_{success}$
      }
      {
        noop
      }}
    \caption{Timeline of Learner Success}
  \end{algorithm}

  \subsection{Result JSON Schema}
\begin{lstlisting}[style=json]
{
    "$schema": "http://json-schema.org/draft-04/schema#",
    "id": "#statement_list",
    "type": "array",
    "items": {"$ref": "#statement"}
}
\end{lstlisting}
  \subsection{Visualization Description}

  description of the associated visualization in english
  \subsection{Visualization prototype}

  This section will be updated to a prototype viz
\end{document}
