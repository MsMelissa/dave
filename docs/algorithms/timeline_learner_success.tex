\documentclass[../main.tex]{subfiles}

\begin{document}

\section{Timeline Of Learner Success}
As learners engage in activities supported by a learning ecosystem, they will build
up a history of learning experiences. When the digital resources of that learning ecosystem
adhere to a framework dedicated to supporting and understanding the
learner, such as the Total Learning Architecture (TLA), it becomes
possible to retell their learning story through data and data visualization. One important aspect of
that story is the' learners history of success.

\subsection{Ideal Statements}
In order to accurately portray a learner's timeline of success, there
are a few base requirements of the data produced by a Learning Record
Provider (LRP). They are as follows:

\begin{itemize}
\item the learner must be uniquely and consistently identified across
  all statements
\item learning activities which evaluate a learner's understanding of material must report if the learner was successful or not
  \begin{itemize}
  \item the grade earned by the learner must be reported
  \item the minimum and maximum possible grade must be reported
  \end{itemize}
\item The learning activities must be uniquely and consistently identified across all statements
\item The time at which a learner completed a learning activity must be recorded
  \begin{itemize}
  \item The timestamp should contain an appropriate level of specificity.
  \item ie. Year, Month, Day, Hour, Minute, Second, Timezone
  \end{itemize}
\end{itemize}

\subsection{Input Data Retrieval}
How to query an LRS via a GET request to the Statements Resource via
curl. The following section contains the appropriate parameters with
example values as well as the curl command necessary for making the
request.\footnote{\label{moreLink} $S$ is the set of all statements
  parsed from the statements array within the HTTP response to the
  Curl request(s). It may be possible that multiple Curl requests are
  needed to retrieve all query results. If multiple requests are
  necessary, $S$ is the result of concatenating the result of each
  request into a single set}\footnote{\label{noZ} Querying an LRS will
  not be defined within the following Z specifications but the results
  of the query will be utilized}\footnote{\label{allTime} If you want
  to query across the entire history of a LRS, omit Since and Until
  from the endpoint(s) and remove the associated \& symbols.}

\begin{lstlisting}[frame=single]
  Agent = "agent={"account":
    {"homePage": "https://example.homepage",
      "name": 123456}}"

  Since = "since=2018-07-20T12:08:47Z"

  Until = "until=2018-07-21T12:08:47Z"

  Base = "https://example.endpoint/statements?"

  endpoint = Base + Agent + "&" + Since + "&" + Until

  Auth = Hash generated from basic auth

  S = curl -X GET -H "Authorization: Auth"
  -H "Content-Type: application/json"
  -H "X-Experience-API-Version: 1.0.3"
  Endpoint
\end{lstlisting}

\subsection{Statement Parameters to Utilize}
The statement parameter locations here are written in
\href{http://goessner.net/articles/JsonPath/}{JSONPath}. This notation
is also compatable with the xAPI Z notation due to the defined
hierarchy of components. Within the Z specifications, a variable name
will be used instead of the $\$$
\begin{itemize}
\item $\$.timestamp$
\item $\$.result.success$
\item $\$.result.score.raw$
\item $\$.result.score.min$
\item $\$.result.score.max$
\item $\$.verb.id$
\end{itemize}

\subsection{2018 Pilot TLA Statement Problems}
The initial pilot test data supports this algorithm.
This section may require updates pending future data review following iterations of the TLA testing.
\subsection{Summary}
\begin{enumerate}
\item Query an LRS via a \href{https://github.com/adlnet/xAPI-Spec/blob/master/xAPI-Communication.md#213-get-statements}{GET} request to the statements endpoint using the parameters agent, since and until
\item Filter the results to the set of statements where:
  \begin{itemize}
  \item $\$.verb.id$ is one of:
    \begin{itemize}
    \item http://adlnet.gov/expapi/verbs/passed
    \item https://w3id.org/xapi/dod-isd/verbs/answered
    \item http://adlnet.gov/expapi/verbs/completed
    \end{itemize}
  \item $\$.result.success$ is true
  \end{itemize}
\item process the filtered data
  \begin{itemize}
  \item extract $\$.timestamp$
  \item extract the score values from $\$.result.score.raw$,
    $\$.result.score.min$ and $\$.result.score.max$ and convert them
    to the scale 0..100
  \item create a pair of [$\$.timestamp$, $\#$]
  \end{itemize}
\end{enumerate}

\subsection{Formal Specification}
\subsubsection{Basic Types}

$COMPLETION$ :== \\ $\{http://adlnet.gov/expapi/verbs/passed\} \, | \\
\{https://w3id.org/xapi/dod-isd/verbs/answered\} \; | \\
\{http://adlnet.gov/expapi/verbs/completed$\} \\
\\
$SUCCESS$ :== $\{true\}$

\subsubsection{System State}

\begin{schema}{TimelineLearnerSuccess}
  Statements \\
  S_{all} : \finset_1 \\
  S_{completion},S_{success},S_{processed} : \finset \\
  \where
  S_{all} = statements \\
  S_{completion} \subseteq S_{all} \\
  S_{success} \subseteq S_{completion} \\
  S_{processed} = \{pair : (statement.timestamp, \nat)\}
\end{schema}
\begin{itemize}

\item The set $S_{all}$ is a non-empty, finite set and is the
  component $statements$
\item The sets $S_{completion}$ and $S_{success}$ are both finite sets
\item the set $S_{completion}$ is a subset of $S_{all}$ which may
  contain every value within $S_{all}$
\item the set $S_{success}$ is a subset of $S_{completion}$ which may
  contain every value within $S_{completion}$
\item the set $S_{processed}$ is a finite set of pairs where each
  contains a $statement.timestamp$ and a natural number
\end{itemize}

\subsubsection{Initial System State}
\begin{schema}{InitTimelineLearnerSuccess}
  TimelineLearnerSuccess \\
  \where
  S_{all} \not = \emptyset \\
  S_{completion} = \emptyset \\
  S_{success} = \emptyset \\
  S_{processed} = \emptyset
\end{schema}
\begin{itemize}
\item The set $S_{all}$ is a non-empty set
\item The sets $S_{completion}$,\,$S_{success}$ and $S_{processed}$ are all initially empty
\end{itemize}

\subsubsection{Filter for Completion}
\begin{schema}{Completion}
  Statement \\
  completion : STATEMENT \pfun \finset \\
  s? : STATEMENT \\
  s! : \finset \\
  \where
  s? = statement \\
  s! = completion(s?) \\
  completion(s?) = \IF s?.verb.id : COMPLETION \\\t5 \THEN s! = s?
  \\\t5 \ELSE s! = \emptyset
\end{schema}
\begin{itemize}
\item The schema $Completion$ inroduces the function $completion$
  which takes in the variable $s?$ and returns the variable $s!$
\item The variable $s?$ is the component $statement$
\item $s!$ is equal to $s?$
  if $\$.verb.id$ is of the type $COMPLETION$ otherwise $s!$ is an empty set
\end{itemize}

\begin{schema}{FilterForCompletion}
  \Delta TimelineLearnerSuccess \\
  Completion \\
  completions : \finset
  \where
  completions \subseteq S_{all} \\
  completions' = \{s : STATEMENT \,|\, completion(s) \not = \emptyset\} \\
  S_{completion}' = S_{completion} \cup completions' \\
\end{schema}
\begin{itemize}
\item the set $completions$ is a subset of $S_{all}$ which may contain
  every value within $S_{all}$
\item The set $completions'$ is the set of all statements $s$ where
  the result of $completion(s)$ is not an empty set
\item the updated set $S'_{completion}$ is the union of the previous
  state of set $S_{completion}$ and the set $completions'$
\end{itemize}

\subsubsection{Filter for Success}
\begin{schema}{Success}
  Statement \\
  success : STATEMENT \pfun \finset \\
  s? : STATEMENT \\
  s! : \finset \\
  \where
  s? = statement \\
  s! = success(s?) \\
  success(s?) = \IF s?.result.success : SUCCESS \\\t4 \THEN s! = s?
  \\\t4 \ELSE s! = \emptyset
\end{schema}
\begin{itemize}
\item the schema $Success$ introduces the function $success$ which
  takes in the variable $s?$ and returns the variable $s!$
\item the variable $s?$ is the component $statement$
\item $s!$ is equal to $s?$ if $\$.result.success$ is of the type
  $SUCCESS$ otherwise $s!$ is an empty set
\end{itemize}

\begin{schema}{FilterForSuccess}
  \Delta TimelineLearnerSuccess \\
  Success \\
  successes : \finset
  \where
  successes \subseteq S_{completion} \\
  successes' = \{s : STATEMENT \,|\, success(s) \not = \emptyset\} \\
  S_{success}' = S_{success} \cup successes'
\end{schema}
\begin{itemize}
\item the set $successes$ is a subset of $S_{completion}$ which may contain
  every value within $S_{completion}$
\item The set $successes'$ contains elements $s$ of type $STATEMENT$
  where $success(s)$ is not an empty set
\item The updated set $S_{success}'$ is the union of the previous
  state of $S_{success}$ and $successes'$
\end{itemize}


\subsubsection{Processes Results}
\begin{schema}{Scale}
  scaled! : \nat \\
  raw?, min?, max? : \num \\
  scale : \num \fun \nat
  \where
  scaled! = scale(raw?, min?, max?) \\
  scale(raw?, min?, max?) = \\\t4
  (raw? * ((0.0 - 100.0) \, div \, (min? - \, max?))) \, + \\ \t4
  (0.0 - (min? * ((0.0 - 100.0) div (min? - \, max?))))
\end{schema}
\begin{itemize}
\item The schema $Scale$ introduces the function $scale$ which takes
  3 arguments, $raw?$, $min?$ and $max?$. The function converts
  $raw?$ from the range $min?..max?$ to 0.0..100.0
\end{itemize}

\begin{schema}{ProcessStatements}
  \Delta TimelineLearnerSuccess \\
  Scale \\
  FilterStatements \\
  processed : \finset
  \where
  processed \subseteq S_{success} \\
  processed' = \{p : (\finset_1\#1 , \nat) \,|\, \\\t3
  \LET \{processed_{i}..processed_{j}\} == \{s_{i}..s_{j}\} @ \\ \t3
  i \leq n \leq j @ \forall s_{n} : s_{i}..s_{j} @
  \exists \, p_{n} : p_{i}..p_{j} @ \\\t3 first~p_{n} = s_{n}.timestamp \, \land
  \\\t3 second~p_{n} = scale(s_{n}.result.score.raw, \\\t7
  s_{n}.result.score.min, \\\t7
  s_{n}.result.score.max)\} \\
  S_{processed}' = S_{processed} \cup processed'

\end{schema}
\begin{itemize}
\item The operation $ProcessStatements$ introduces the variable
  $processed$ which is a subset of $S_{success}$ which may contain
  every value within $S_{success}$
\item $S_{success}$ is the result of the operation $FilterStatements$
\item The operation defines the variable $processsed'$ which is a
  set of objects $p$ which are ordered pairs of (1) a finite set
  containing one value and (2) a single positive number.
\item The first component of every object $p$, is the
  timestamp from the associated $statement$ within $processed$
  ie. $s.timestamp$
\item The second component of every object $p$ is the result
  of the function $scale$. The score values contained within the
  associated $statement$ $s$ are the arugments passed to $scale$. ie $scale(s.result.score.raw, s.result.score.min,
  s.result.score.max)$
\item The result of the operation $ProcessStatements$ is to updated
  the set $S_{processed}$ with the values contained within $processed'$
\end{itemize}

\subsubsection{Sequence of Operations}

$FilterStatements \defs FilterForCompletion \semi FilterForSuccess$
\begin{itemize}
\item The schema $FilterStatements$ is the sequential composition
  of operation schemas $FilterForCompletion$ and
  $FilterForSuccess$
\item $FilterForCompletion$ happens before $FilterForSuccess$
\end{itemize}
%%% this fixes indenting
$ProcessedStatements \defs FilterStatements \semi ProcessStatements$
\begin{itemize}
\item The schema $ProcessedStatements$ is the sequential composition
  of operation schemas $FilterStatements$ and
  $ProcessStatements$
\item $FilterStatements$ happens before $ProcessStatements$
\end{itemize}

\subsubsection{Return}
\begin{schema}{Return}
  \Xi TimelineLearnerSuccess \\
  ProcessedStatements \\
  S_{processed}! : \finset
  \where
  S_{processed}! = S_{processed}
\end{schema}
\begin{itemize}
\item The returned variable $S_{processed}!$ is equal to the current
  state of variable $S_{processed}$ after the operations
  $FilterForCompletion$, $FilterForSuccess$ and $ProcessStatements$
\end{itemize}

\subsection{Pseudocode}

\begin{algorithm}[H]
  \SetAlgoLined
  \KwIn{$S_{all}$}
  \KwResult{$coll'$}
  \emph{coll = []}\;
  \While{$S_{all} \not = \emptyset$}
  {\ForEach{$s \in S_{all}$}
    {\eIf{$s.verb.id = COMPLETION$}
      {\bf do \\
        $S_{completion}' \leftarrow s \cup S_{completion}$\;
        $S_{all}' \leftarrow S_{all} \setminus s$ \;
        recur $S_{completion}', S_{all}'$\;}
      {\bf do \\
        $S_{all}' \leftarrow S_{all} \setminus s$\;
        recur $S_{all}'$\;}}}
  \While{$S_{completion}' \not = \emptyset$}
  {\ForEach{$sc \in S_{completion}'$}
    {\eIf{$sc.result.success = SUCCESS$}
      {\bf do \\
        $S_{success}' \leftarrow sc \cup S_{success}$\;
        $S_{completion}' \leftarrow S_{completion} \setminus sc$\;
        recur $S_{success}', S_{completion}'$\;}
      {\bf do \\
        $S_{completion}' \leftarrow S_{completion} \setminus sc$\;
        recur $S_{completion}'$\;}}}
  {\ForEach{$ss \in S_{success}'$}
    {$raw? \leftarrow ss.result.score.raw$\;
     $max? \leftarrow ss.result.score.max$\;
     $min? \leftarrow ss.result.score.min$\;
     $scaled \leftarrow scale(raw?, min?, max?)$\;
     $subVec \leftarrow [ss.timestamp, scaled]$\;
     $coll' \leftarrow coll \cat subVec$\;
     {\bf recur $coll'$}}
   \Return $coll'$}
  \caption{Timeline of Learner Success}
\end{algorithm}
\begin{itemize}
\item The Z schemas are used within this pseudocode
\item The return value coll is an array of arrays, each containing a
  $statement.timestamp$ and a scaled score.
\end{itemize}

\subsection{JSON Schema}
\begin{lstlisting}[]
{"type":"array",
   "items":{"type":"array",
      "items":[{"type":"string"}, {"type":"number"}]}}
\end{lstlisting}

\subsection{Visualization Description}

The \textbf{Timeline of Learner Success} visualization will be a line chart
where the domain is time and the range is score on a scale of 0.0 to
100.0. Every subarray will be a point on the chart. The domain of the graph should be in
chronological order. \\

\subsection{Visualization prototype}
\pgfplotstabletypeset[string type]

\begin{tikzpicture}
  \begin{axis}[
    title = Timeline of Learner Success,
    ylabel = $score(\%)$,
    ymin = 0,
    ymax = 100,
    date coordinates in=x,
    xtick=data,
    xticklabel style=
    {rotate=90,anchor=near xticklabel},
    xticklabel=\month:\day. \hour:\minute,]
    \addplot coordinates {
      (2018-07-15 22:22, 75.0)
      (2018-07-16 15:15, 84.0)
      (2018-07-17 18:18, 81.0)
      (2018-07-18 13:13, 91.0)
      (2018-07-19 17:17, 88.0)
      (2018-07-20 06:06, 85.0)
      (2018-07-21 12:12, 87.0)
    };
  \end{axis}
\end{tikzpicture}

\subsection{Prototype Improvement Suggestions}
Additional features may be implemented on top of this base
specification but they would require adding aditional values to each
subarray returned by the algorithm. These additional values can be
retrieved via (1) performing metadata lookup within or independently of the
algorithm (2) by utilizing additional xAPI statement paramters and/or (3) by
performing additional computations. The following examples assume the
metadata is contained within each statement available to the algorithm.
\begin{itemize}
\item A tooltip containing the name of an activity when hovering
  over a specific point on the chart
  \begin{itemize}
  \item this would require utilizing $\$.object.definition.name$
  \end{itemize}
\item A tooltip containing the device on which the activity was experienced
  \begin{itemize}
  \item this would require utilizing $\$.context.platform$
  \end{itemize}
\item A tooltip containing the instructor associated with a
  particular data point
  \begin{itemize}
  \item this would require utilizing $\$.context.instructor$
  \end{itemize}
\end{itemize}

\end{document}
