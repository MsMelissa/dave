%%% Local Variables:
%%% mode: latex
%%% TeX-master: t
%%% End:

\documentclass{article}

\usepackage[ruled]{algorithm2e}
\usepackage{hyperref}
\usepackage{amsmath}
\usepackage{zed-csp}
\usepackage{breqn}
\usepackage{listings}
\lstset{literate = {-}{-}1} % get dashs to show up
%%% TODO: get items to respect latex
\begin{document}
\section*{Timeline Of Learner Success}
As learners engage in a blended eLearning ecosystem, they will build up a history of learning experiences. When that eLearning ecosystem adheres to a framework dedicated to supporting and understanding the learner, such as the Total Learning Architecture (TLA), it becomes possible to retell their story through data. One important aspect of that story is the learner's history of success.

\section{Ideal Statements}
In order to accurately portray a learner's timeline of success, there are a few requirements of the data produced by a Learning Record Provider (LRP). They are as follows:
\begin{itemize}
\item the learner must be uniquely and consistently identified across all LRPs
\item learning activities which access a learner's understanding of material should report if the learner was successful or not
  \begin{itemize}
  \item if the assessment is scored, the grade earned by the learner should be reported
  \item if the assessment is scored, the minimum and maximum possible grade should be reported
  \end{itemize}
\item The learning activities must be uniquely and consistently identified across all LRPs
\item The time at which a learner completed a learning activity must be recorded
  \begin{itemize}
  \item The timestamp should contain an appropriate level of specificity.
  \item ie. Year, Month, Day, Hour, Minute, Second, Timezone
  \end{itemize}

\end{itemize}

\subsection{statement parameters to utilize}
The statement parameter locations here are written in \href{http://goessner.net/articles/JsonPath/}{JSONPath}
\begin{itemize}
  \item \$.timestamp
  \item \$.result.success
  \item \$.actor
  \item \$.verb.id
  \end{itemize}


\section{TLA Statement problems}
The data collected at the TLA pilot run supports the following algorithm.
\section{Algorithm}
\subsection{Summary}
\begin{enumerate}
  \item Query an LRS via a \href{https://github.com/adlnet/xAPI-Spec/blob/master/xAPI-Communication.md#213-get-statements}{GET} request to the statements endpoint using the parameters agent, since and until
  \item Filter the results to the set of statements where:
    \begin{itemize}
    \item \$.verb.id is one of:
      \begin{itemize}
      \item http://adlnet.gov/expapi/verbs/passed
      \item https://w3id.org/xapi/dod-isd/verbs/answered
      \item http://adlnet.gov/expapi/verbs/completed
      \end{itemize}
    \item \$.result.success is true
    \end{itemize}
  \end{enumerate}

  \subsection{Query an LRS via REST}
  How to query an LRS via a GET request to the Statements Resource
  \footnote{\label{moreLink} S is the set of all statements parsed from the statements array within the HTTP response to the Curl request. It may be possible that multiple Curl requests are needed to retrieve all query results. If multiple requests are necessary, S is the result of concatenating the result of each request into a single set}
  \begin{lstlisting}[frame=single]
Agent = "agent={"account":
                {"homePage": "https://example.homepage",
                 "name": 123456}}"

Since = "since=2018-07-20T12:08:47Z"

Until = "until=2018-07-21T12:08:47Z"

Base = "https://example.endpoint/statements?"

endpoint = Base + Agent + "&" + Since + "&" + Until

Auth = Hash generated from basic auth

S = curl -X GET -H "Authorization: Auth"
         -H "Content-Type: application/json"
         -H "X-Experience-API-Version: 1.0.3"
         Endpoint
  \end{lstlisting}

  \subsection{Z Specifications}
  \subsubsection{xAPI Schema}
  [Statement]
  [Actor]
  [Verb]
  [Object]
  [Result]
  [Context]
  [Timestamp]
  \begin{schema}{Statement}
    s : Statement
    \where
    s = \{Actor, Verb, Object, Timestamp\} \, |
    \\ \t1 \{Actor, Verb, Object, Timestamp, Context\} \, |
    \\ \t1 \{Actor, Verb, Object, Timestamp, Result\} \, |
    \\ \t1 \{Actor, Verb, Object, Timestamp, Result, Context\}
  \end{schema}
  \begin{itemize}
  \item The variable s is of type Statement and consists of an Actor, Verb, Object, Timestamp and optionally Context and Result
  \end{itemize}
  \begin{schema}{Statements}
    S : Statements
    \where
    S = \{~s : Statement \, | \, S \, \neg \, \emptyset\} \\
  \end{schema}
  \begin{itemize}
  \item The variable S is of type Statements and is a set of objects s, each of type Statement
  \item The variable S is a non empty set
  \end{itemize}

  \subsubsection{Timeline Leaner Success System State}
  \begin{schema}{TimelineLearnerSuccess}
    S_{extra},S_{completion},S_{success},S_{failure} : \power S \\
    \where
    S_{extra} \cup S_{completion} = S \\
    S_{extra} \cap S_{completion} = \{\} \\
    S_{success} \cup S_{failure} = S_{completion} \\
    S_{success} \cap S_{failure} = \{\}
  \end{schema}
  \begin{itemize}
  \item The sets S_{extra},\,S_{completion},\,S_{success},\,S_{failure} are the powerset of S
  \item The union of sets S_{extra} and S_{completion} is equal to the complete set of statements S
  \item No values are shared between the sets S_{extra} and S_{completion}
  \item The union of sets S_{success} and S_{failure} is equal to the set S_{completion}
  \item No values are shared between the sets S_{success} and S_{failure}
  \end{itemize}

  \subsubsection{Initial State of Timeline Learner Success System}
  \begin{schema}{InitTimelineLearnerSuccess}
    TimelineLearnerSuccess \\
    \where
    S_{extra} = \{\} \\
    S_{completion} = \{\} \\
    S_{success} = \{\} \\
    S_{failure} = \{\} \\
  \end{schema}
  \begin{itemize}
  \item The sets S_{extra},\,S_{completion},\,S_{success},\,S_{failure} are all initially empty
  \end{itemize}

  \subsubsection{Filter for Completion}
  \begin{schema}{VerbIdCompletion}
    V_{completion} : Verb
    \where
    V_{completion} = http://adlnet.gov/expapi/verbs/passed \; | \\
    \t3 https://w3id.org/xapi/dod-isd/verbs/answered \; | \\
    \t3 http://adlnet.gov/expapi/verbs/completed
  \end{schema}
  \begin{itemize}
    \item The var V_completion has a value of one of the above IRIs and is of type Verb
  \end{itemize}
  \begin{schema}{FilterForCompletion}
    \Delta TimelineLearnerSuccess \\
    \where
    S_{completion}' = \{~s : Statement \, | \, V_{completion} \in s \, \land \, s \in S \} \\
    S_{extra}' = \{~s : Statement \, | \, V_{completion} \not \in s \, \land \, s \in S \}
  \end{schema}
  \begin{itemize}
  \item The updated set S'_{completion} is the set of all statements s where V_{completion} is in s and s is in S
  \item the updated set S'_{extra} is the set of all statements s where V_{completion} is not in s and s is in s
  \end{itemize}

  \subsubsection{Filter for Success}
  \begin{schema}{ResultSuccessTrue}
    R_{successful} : Result
    \where
    R_{successful} = true \\
    R_{successful} \not = false
  \end{schema}
  \begin{itemize}
    \item The var R_{successful} has a value of true but not false and is of the type Result
  \end{itemize}

  \begin{schema}{FilterForSuccess}
    \Delta TimelineLearnerSuccess \\
    s_{completion} : Statement
    \where
    s_{completion} \in S_{completion} \\
    S_{success}' = \{~s_{completion} : Statement \, | \, R_{successful} \in s_{completion}\} \\
    S_{failure}' = \{~s_{completion} : Statement \, | \, R \not \in s_{completion}\}
  \end{schema}
  \begin{itemize}
  \item The set s_{completion} is of type Statement and is in the set S_{completion}
  \item The updated set S_{success}' is the set of all statements s_{completion} where R_{successful} is in s_{completion}
  \item The updated set S_{failure}' is the set of all statements s_{completion} where R_{successful} is not in s_{completion}
  \end{itemize}

  \subsubsection{Return}
  \begin{schema}{Return}
    \Xi TimelineLearnerSuccess \\
    S_{success}! : Statements
    \where
    S_{success}! = S_{success}
  \end{schema}
  \begin{itemize}
    \item The returned variable S_{success}! is equal to the current state of variable S_{success}
  \end{itemize}

  \subsection{Pseudocode}
  \footnote{\label{variableDescription} S_{completion} = S-completion s_{completion} = s-completion, S_{extra} = S-extra, V_{completion} = V-completion,
     R_{successful} = R-successful, S_{success} = S-success, S_{failure} = S-failure}
  \begin{algorithm}[H]
    \SetAlgoLined
    \KwIn{S}
    \KwResult{S-success}
    \While{S is not empty} {
      for each Statement s in S \\
      \eIf {s.verb.id = V-completion}
      {
        add s to S-completion
      }
      {
        add s to S-extra
      }}
    \While {S-completion is not empty} {
      for each Statement s-completion in S-completion \\
      \eIf {s-completion.result.success = R-success}
      { add s-completion to S-success }
      { add s-completion to S-failure }
    }
  \end{algorithm}
  \subsection{Result JSON Schema}
  \subsection{Visualization Description}
  description of the associated visualization in english
  \subsection{VEGA example}
  This section will be updated to include a VEGA JSON blob for prototype viz
\end{document}
