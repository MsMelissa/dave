\documentclass[../../main.tex]{subfiles}

\begin{document}

\section{[Primitive Name]}

\begin{itemize}

\item introduces the primitive - short narative describing the operation

\end{itemize}

\subsection{Input data}

\begin{itemize}

\item describes the expectations of the input data (shape/type/number/etc)

\item possible subsubsections for each generic category of expectation

  \begin{itemize}
  \item one of these SHOULD be a JSON Schema or other type of machine readable description of the data
  \end{itemize}

\end{itemize}

\subsection{Output Data}

\begin{itemize}

\item describes the expectations of the output data (shape/type/number/etc)

\item possible subsubsections for each generic category of expectation

  \begin{itemize}
  \item one of these SHOULD be a JSON Schema or other type of machine readable description of the data
  \end{itemize}

\end{itemize}

\subsection{Operation}

\begin{itemize}

\item summary of operation in numbered list of steps

\end{itemize}

\subsubsection{Formal Specification}


\begin{itemize}

\item z spec for the primitive

\end{itemize}

\subsubsection{Pseudocode}

\begin{itemize}

\item should follow the format used in the existing Pseudocode sections of each algorithm definition

\end{itemize}

\section{Recommended Primitives}

\begin{itemize}

\item intro text for the purpose of this section

\end{itemize}

\subsection{Complementary}

\begin{itemize}

\item primitives which should be used hand-in-hand with the primitive defined in this tex document

  \begin{itemize}
  \item doesn't consider ordering
  \end{itemize}

\end{itemize}

\subsection{Input Data Producers}

\begin{itemize}

\item primitives which produce output data which can serve as input data to the primitive defined in this tex document

\item subset of the primitives referenced within the Complementary section

\end{itemize}

\subsection{Output Data Consumers}

\begin{itemize}

\item primitives which consume the output data of the primitive defined in this tex document

\item subset of the primitives referenced within the Complementary section

\end{itemize}

\subsection{Alternatives}

\begin{itemize}

\item primitives which may be used instead of the primitive defined in this tex document

\item subset of the primitives referenced within the Complementary section

\end{itemize}

\subsection{Related}

\begin{itemize}

\item primitives which are in some other way related to the primitive defined in this tex document

\item MAY be a subset of the primitives referenced wtihin the Complementary section

\end{itemize}

\subsubsection{By Operation}

\begin{itemize}

\item references to other Primitives which have a similar but not identical operation

\end{itemize}

\subsubsection{By Input Data}

\begin{itemize}

\item references to other Primitives which expect similar or idential input data

\end{itemize}

\subsubsection{By Output Data}

\begin{itemize}

\item references to other Primitives which produce similar output data

\end{itemize}

\end{document}
