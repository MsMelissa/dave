\documentclass[../main.tex]{subfiles}

\begin{document}
There will be many Primitives used within Algorithm definitions in DAVE but
navigation into a nested $Collection$ or $KV$ is most likely to be used
across nearly all Algorithm definitions. In the following section,
helper Operations are introduced for navigation into and back out of a
nested Value. These Operations are then used to define the common Primitives centered around
traversal of nested data structures ie. xAPI Statements and Algorithm State.

\subsection{Traversal Operations}
\begin{schema}{Get[V, Collection]}
  in?, v! : V \\
  id? : Collection \\
  get~\_ : V \cross Collection \surj V
  \where
  v! = get(in?, id?) @\\
  \t1 \ ~  = (atIndex(in?, head(id?)) \iff (array?(in?) = true) ~\land~ (head(id?) \in \nat)) ~\lor \\
  \t1 \ \ \ \ ~ (atKey(in?, head(id?)) \iff (array?(in?) = false) ~ \land ~ (map?(in?) = true))
\end{schema}
\begin{itemize}
  \item retrieval of a $V$ located at $id?$ within $in?$ where $in?$ can be a $Collection$ or $KV$
\end{itemize}
\begin{schema}{Merge[(V, V), Collection]}
  parent?, child?, parent! : V \\
  at? : Collection \\
  merge~\_ : (V \cross V) \cross Collection \bij V
  \where
  parent! = merge((parent?, child?), at?) @ \\
  \t1 \ \ \ \ ~ = (associate(parent?, head(at?), child?) \\
  \t3 \iff map?(parent?) = true) ~ \lor \\
  \t2 \ ~ (update(parent?, child?, head(at?)) \\
  \t3 \iff (array?(parent?) = true) ~\land~ (head(at?) \in \nat))
\end{schema}
\begin{itemize}
  \item Updating of $parent?$ to include $child?$ at location indicated by $head(at?)$
\end{itemize}

\begin{schema}{Conj[V, V]}
  parent?, data? : V \\
  conj! : Collection \\
  conj~\_ : V \cross V \bij Collection
  \where
  conj! = conj(parent?, data?) @ \\
  \t3 let \ ~ j == first(last(parent?)) \\
  \t4 parent?_{coll} == append(\langle  \rangle, parent?, 0) \\
  \t1 \ = (append(~parent?~, ~data?, (j + 1)) \iff array?(parent?) = true) ~ \lor \\
  \t1 \ \ \ \ (append(parent?_{coll}, ~data?, (j + 1)) \iff array(parent?) = false)
\end{schema}
\begin{itemize}
  \item $conj!$ is a collection with $data?$ at the last index $conj!_{j} = data?$.
\end{itemize}
\subsection{Traversal Primitives}
The helper Operations defined above are used to describe the traversal of a heterogeneous nested Value.
In the following subsections, examples which demonstrate the functionality of Primitives will be passed $X$ as $in?$.
\begin{argue}
  X = \langle x_{0}, x_{1}, x_{2} \rangle \\
  \t1 x_{0} = true \\
  \t1 x_{1} = \langle a, b, c \rangle \\
  \t1 x_{2} = \ldata foo \mapsto \ldata bar \mapsto buz, x \mapsto y , z \mapsto \langle 3, 2, 1 \rangle \rdata \rdata \\
  fn! = fn(X_{\langle path?_{i}~..path?_{j-1} \rangle}, v?) @ \\
  \t2 \forall X_{\langle path?_{i}~..path?_{j-1} \rangle} ~\land~ v? ~|~ fn! = ZZZ & always return $ZZZ$ \\
\end{argue}

\subsubsection{Get In}
Collection and KV have different Fundamental Operations for navigation into, value extraction from
and application of updates to. Navigation into an arbitrary Value without concern
for its type is a useful tool to have and has been defined as the Primitive $getIn$.
\begin{schema}{GetIn[V, Collection]}
  Get, Recur \\
  in?, atPath! : V \\
  path? : Collection \\
  getIn~\_ : V \cross Collection \surj V
  \where
  getIn = \langle get~\_~, recur~\_ \rangle \bsup \#~path? - 1 \esup \\ ~ \\
  atPath! = getIn(in?, path?) @ \\
  \ \forall n : i~..~j-1 @ j = first(last(path?)) \implies first(j, path?_{j}) ~|~ \exists ~ down_{n} @\\
  \t1 let \ \ path?_{n} == tail(path?)\bsup n-i \esup \\
  \t1 \ \ \ \ \ down_{i} == get(in?, path?_{n}) \implies \\
  \t5 atIndex(in?, head(path?)) ~\lor \\
  \t5 atKey(in?, head(path?)) \iff n = i\\
  \t1 \ \ \ \ \ down_{n} == recur(down_{i}, path?_{n}, get~\_)\bsup j - 1 \esup \\
  \t1 \ \ \ \ \ down_{j-1} == get(down_{n}, path?_{n}) \iff n = j - 2 \\ ~ \\
  atPath! = down_{j} = get(down_{j-1}, path?_{n}) @ \\
  \t5 \ path?_{n} \equiv (path? \extract j) \implies \\
  \t6 \ \langle j \mapsto atIndex(path?, j) \rangle \iff n = j-1 \\
\end{schema}
The following examples demonstrate the functionality of the Primitive $getIn$
\begin{argue}
  getIn(X, \langle 1, 1 \rangle) = b \\
  getIn(X, \langle 0 \rangle) = true \\
  getIn(X, \langle 2, foo, z, 0 \rangle) = 3
\end{argue}
% FIXME: wordsmith...sounds like english translation of Z schema. Can be said in a more clear/concise way.
Additionally, the propagation of an update, starting at some depth within a passed in Value and bubbling up to the top level,
such that the update is only applied to values along a specified path as necessary, is also a useful tool to have.
The following sections introduce Primitives which address performing these types of updates and ends with a summary of
the functional steps described in the sections below. $replaceAt$ is introduced first and serves as a point of comparison
when describing the more abstract Primitives $backProp$ and $walkBack$.

\subsubsection{Replace At}
The schema $ReplaceAt$ uses the helper Operation $merge$ to apply updates while climbing up from some arbitrary depth.
\begin{schema}{ReplaceAt[V, Collection, V]}
  GetIn, Merge \\
  in?, with?, out! : V \\
  path? : Collection \\
  replaceAt~\_ : V \cross Collection \cross V \bij V
  \where
  replaceAt = \langle \langle getIn~\_~, ~merge~\_~ \rangle , ~recur~\_ \rangle \bsup \#~path? - 1 \esup \\ ~ \\
  out! = replaceAt(in?, path?, with?) @ \\
  \t1 \forall n : i~..~j-1 @ (i = first(head(path?))) ~\land~ (j = first(last(path?))) ~|~ \exists ~ parent_{n} @ \\
  \t3 let \ ~ path?_{n} == tail(path?)\bsup n-i \esup \\
  \t2 parent_{n} = recur(parent_{n-1}, path?_{n}, get~\_)\bsup j - 1 \esup \implies \\
  \t3 let \ ~ parent_{i} == getIn(in?, path?_{n}) \iff n = i \\
  \t3 \ \ \ \ ~ parent_{i+1} == getIn(parent_{i}, path?_{n}) \iff n = i+1 \\
  \t3 \ \ \ \ ~ parent_{j-1} == getIn(parent_{j-2}, path?_{n}) \iff n = j-1 \\
  \t3 parent_{j} =  getIn(parent_{j-1}, (path? \extract j))
  \\ ~ \\
  \t1 \forall z : p~..~q @ (p = j-1) ~\land ~ (q = i + 1) \implies \\
  \t4 ((z = p \iff n = j-1) \land (z = q \iff n = i + 1)) ~|~ \exists ~ child_{z} @ \\
  \t3 let \ \ path?_{rev} == rev(path?) \\
  \t3 \ \ \ \ \ path?_{z} == tail(path?_{rev})\bsup p-z+1 \esup \\
  \t2 child_{z} = recur((parent_{n}, child_{n+1}), path?_{z}, merge~\_ ) \\
  \t3 let \ ~ child_{p} == merge((parent_{n}, with?), path?_{z}) \iff z = p \implies n = j-1 \\
  \t3 \ \ \ \ ~ child_{p+1} == merge((parent_{n}, child_{p}), path?_{z}) \iff n = j - 2 ~\land ~p = j - 1 \\
  \t3 \ \ \ \ ~ child_{q} == merge((parent_{n}, child_{q+1}), path?_{z}) \iff z = q \implies n = i + 1
  \\~\\
  out! = merge((in?, child_{q}), path?_{n}) \equiv merge((in?, child_{q}), (path? \extract i)) \iff (n = i = q - 1)
\end{schema}
\begin{itemize}
\item The range of indices $i~..~j-1$ is used to describe navigation into some Value given $path?$
  \begin{itemize}
  \item Used to reference preceding level of depth
  \item keeps track of parent from previous steps
  \end{itemize}
\item The range of indices $p~..~q$ is used to describe navigation up from target depth indicated by $path?$
  \begin{itemize}
  \item Used to reference current level of depth
  \item keeps track of child after the update has been applied
  \end{itemize}
\item The propagation of the update starts with $child_{p}$
  \begin{itemize}
  \item $with?$ is added to $parent_{j-1}$ at $get~(path?, \langle ~j~ \rangle)$
  \item parent nodes need to be notified of the change within their children
  \end{itemize}
\end{itemize}
The following examples demonstrate the functionality of the Primitive $replaceAt$
\begin{argue}
  replaceAt(X, \langle 2, foo, q \rangle, fn!) = \langle x_{0}, x_{1}, \ldata foo \mapsto \ldata bar \mapsto buz, x \mapsto y, q \mapsto ZZZ \rdata \rdata\rangle \\
  replaceAt(X, \langle 2, foo, x \rangle, fn!) = \langle x_{0}, x_{1}, \ldata foo \mapsto \ldata bar \mapsto buz, x \mapsto ZZZ \rdata \rdata\rangle \\
\end{argue}
This Primitive can be made more general purpose by replacing $merge$ with a placeholder $fn?$
representing a passed in Operation or Primitive.

\subsubsection{Back Prop}
Being able to pass a function as an argument allows for, in this context, the arbitrary handling of
how update(s) are applied at each level of nesting. The arbitrary $fn?$ should expect
a (Parent, Child) tuple and a Collection of indices as arguments and return a potentially modified version of the parent.
\begin{schema}{BackProp~[V, Collection, V, (~\_\pfun~\_~)]}
  GetIn \\
  in?, fnSeed?, out! : V \\
  path? : Collection \\
  fn? : (~\_\pfun~\_~) \\
  backProp~\_ : V \cross Collection \cross V \cross (~\_\pfun~\_~) \bij V
  \where
  backProp = \langle \langle getIn~\_~, fn?~\_ \rangle , ~recur~\_ \rangle \bsup \#~path? - 1 \esup \\ ~ \\
  out! = backProp~(in?, path?, fnSeed?, fn?) @ \\
  \t1 \forall n : i~..~j-1 @ (i = first(head(path?))) ~\land~ (j = first(last(path?))) ~|~ \exists ~ parent_{n} @ \\
  \t3 let \ ~ path?_{n} == tail(path?)\bsup n-i \esup \\
  \t2 parent_{n} = recur(parent_{n-1}, path?_{n}, get~\_)\bsup j - 1 \esup \implies \\
  \t3 let \ ~ parent_{i} == getIn(in?, path?_{n}) \iff n = i \\
  \t3 \ \ \ \ ~ parent_{i+1} == getIn(parent_{i}, path?_{n}) \iff n = i+1 \\
  \t3 \ \ \ \ ~ parent_{j-1} == getIn(parent_{j-2}, path?_{n}) \iff n = j-1 \\
  \t3 parent_{j} =  getIn(parent_{j-1}, (path? \extract j))
  \\ ~ \\
  \t1 \forall z : p~..~q @ (p = j-1) ~\land ~ (q = i + 1) \implies \\
  \t4 ((z = p \iff n = j-1) \land (z = q \iff n = i + 1)) ~|~ \exists ~ child_{z} @ \\
  \t3 let \ \ path?_{rev} == rev(path?) \\
  \t3 \ \ \ \ \ path?_{z} == tail(path?_{rev})\bsup p-z+1 \esup \\
  \t2 child_{z} = recur((parent_{n}, child_{n+1}), path?_{z}, fn?) \\
  \t3 let \ ~ child_{p} == fn?((parent_{n}, fnSeed?), path?_{z}) \iff z = p \implies n = j-1 \\
  \t3 \ \ \ \ ~ child_{p+1} == fn?((parent_{n}, child_{p}), path?_{z}) \iff n = j - 2 ~\land ~p = j - 1 \\
  \t3 \ \ \ \ ~ child_{q} == fn?((parent_{n}, child_{q+1}), path?_{z}) \iff z = q \implies n = i + 1
  \\~\\
  out! = fn?((in?, child_{q}), path?_{n}) \equiv fn?((in?, child_{q}), (path? \extract i)) \iff (n = i = q - 1)
\end{schema}
The schema $ReplaceAt$ was introduced before $BackProp$ so the process underlying both
could be explicitly demonstrated and defined. The hope is that this made the introduction of
the more abstract Primitive $backProp$ easier to follow. A quick comparison of $ReplaceAt$ and
$BackProp$ reveals that the only major difference between them is $fn?$ vs $merge~\_$.
This implies the Primitive $backProp$ can be used to replicate $replaceAt$.
\begin{zed}
  replaceAt(in?, path?, with?) \equiv \\
  \t1 backProp~(in?, path?, fnSeed?, merge~\_) \iff with? = fnSeed?
\end{zed}
Above highlights the arguments $with? ~\land~ fnSeed?$ which serve the same purpose within $backProp$ and $replaceAt$.
\begin{itemize}
\item Within $ReplaceAt$, the naming $with?$ indicates its usage with respect to $merge$ and the overall functionality of the Primitive
\item Within $BackProp$, the naming $fnSeed?$ indicates that the usage of the variable within $fn?$ is unknowable but this value
will be passed to $fn?$ on the very first iteration of the Primitive
\end{itemize}
In both cases, the variable is put into a tuple and passed to $fn?$.
\begin{zed}
  backProp(X, \langle 2, foo, x \rangle, fn!, merge~\_) = \langle x_{0}, x_{1}, \ldata foo \mapsto \ldata bar \mapsto buz, x \mapsto ZZZ \rdata \rdata\rangle \\
\end{zed}
The notable limitation of $backProp$ are enumerated in the bullets bellow and the Primitive $walkBack$ is introduced to address them.
\begin{itemize}
\item expectation of a seeding value ($fnSeed?$) as a passed in argument
\item the general dismissal of the value ($parent_{j}$) located at $path?$ which is potentially being overwritten
\end{itemize}

\subsubsection{Walk Back}

In the Primitive $walkBack$, $fnSeed?$ is assumed to be the result of a function $fn?_{\delta}$
which is passed in as an argument. $fn?_{\delta}$ will be passed $parent_{j}$ as an argument in order to produce $fnSeed?$.
This Value will then be used exactly as it was in $backProp$ given $walkBack$ expects another function argument $fn?_{nav}$.
\begin{zed}
  walkBack(in?, path?, fn?_{\delta}, fn?_{nav})
\end{zed}
In fact, the usage of $fn?_{nav}$ in $WalkBack$ is exactly the same as the usage of $fn?$
in $BackProp$ as $fn?_{nav}$ is passed to $backProp$ as $fn?$.

\begin{schema}{WalkBack[V, Collection, (~\_~\pfun~\_~), (~\_~\pfun~\_~)]}
  BackProp \\
  in?, out! : V \\
  path? : Collection \\
  fn?_{\delta}, fn?_{nav} : (~\_~\pfun~\_~) \\
  walkBack~\_ : V \cross Collection \cross (~\_~\pfun~\_~) \cross (~\_~\pfun~\_~) \bij V
  \where
  walkBack = \langle getIn~\_ ~, ~fn?_{\delta}~\_~, ~backProp~\_ \rangle \\ ~ \\
  out! = walkBack(in?, path?, fn?_{\delta}, fn?_{nav}) @ \\
  \t1 let \ ~ fnSeed == fn?_{\delta}(getIn(in?, path?)) \\
  \t1 = backProp(in?, path?, fnSeed, fn?_{nav})
\end{schema}
By replacing $fnSeed?$ with $fn?_{\delta}$ as an argument
\begin{itemize}
\item $walkBack$ can be used to describe predicate based traversal of $in?$
\item $walkBack$ can be used to update Values at arbitrary nesting within $in?$ and at the same time describe how those changes affect the rest of $in?$
\end{itemize}
$walkBack$ serves as a graph traversal template Primitive whose behavior is defined in terms of the nodes within $in?$
and the interpretation of those nodes via $fn?_{\delta}$ and $fn?_{nav}$. This establishes the means for defining Primitives which can make
longitudinal updates as needed before making horizontal movements through some $in?$. In order for $backProp$ to be used in the same way, the required state must be managed by
\begin{itemize}
\item $fn_{nav}$
\item some higher level Primitive that contains $backProp$ (see $WalkBack$)
\end{itemize}
This important difference means $walkBack$ can be used to replicate $backProp$ but the opposite is not always true.
\begin{zed}
  walkBack(in?, path?, fn?_{\delta}, fn?_{nav}) \equiv \\
  \t1 backProp(in?, path?, fnSeed?, fn?_{nav}) \iff fnSeed? = fn?_{\delta}(getIn(in?, path?))
\end{zed}
This means $replaceAt$ can also be replicated.
\begin{zed}
  replaceAt(in?, path?, with?) \equiv \\
  \t1 (backProp~(in?, path?, fnSeed?, merge~\_) \iff with? = fnSeed?) \equiv \\
  \t2 walkBack(in?, path?, fn?_{\delta}, merge~\_) \iff \\
  \t3 fn?_{\delta}(getIn(in?, path?)) = fnSeed? = with?
\end{zed}
The following examples demonstrate the functionality of $walkBack$
\begin{argue}
  walkBack(X, \langle 0 \rangle, array?~\_~, merge~\_) = \langle false, x_{1}, x_{2} \rangle \\
  walkBack(X, \langle 2, qux \rangle, fn~\_~, merge~\_) = \langle x_{0}, x_{1}, (x_{2} ~\cup~ qux \mapsto ZZZ)\rangle \\
  walkBack(X, \langle 1, 0 \rangle, succ~\_, merge~\_) = \langle x_{0}, \langle b, b, c \rangle, x_{2} \rangle
\end{argue}

\subsection{Summary}
The following is a summary of the general process which has been described in the previous sections.
The variable names here are NOT intended to be 1:1 with those in the formal definitions (but
there is some overlap) and the summary utilizes the Traversal Operations defined at the start of the section.
\begin{enumerate}
\item navigate down into the provided value $in?$ up until the second to last value $in?_{path?_{j-1}}$ as described by the provided $path?$
  \begin{zed}
    in?_{path?_{j-1}} : V
    \where
    path?_{j-1} \implies path? ~\ndres~ j \implies path? \dres ~(~\dom ~path? ~\setminus ~\{j\})
  \end{zed}
\item extract any existing data mapped to $atIndex(path?, j)$ from the result of step 1
  \begin{zed}
    in?_{path?} : V
    \where
    path? \implies path?_{j-1} ~\cup~ (j, atIndex(path?, j))
  \end{zed}
\item create the mapping $(atIndex(path?, j), in?_{path?})$ labeled here as $args?$
  \begin{zed}
    args? = (atIndex(path?, j), in?_{path?})
    \where
    args? \in in?_{path?_{j-1}}\\
    first(args?) = atIndex(path?, j)
  \end{zed}
\item pass $in?_{path?}$ to the provided function $fn?$ to produce some output $fn!$
  \begin{zed}
    fn! = fn?(second(args?)) = fn?(in?_{path?})
   \end{zed}
\item replace the previous mapping $args?$ within $in?_{path?_{j-1}}$ with $fn!$ at $atIndex(path?, j)$
  \begin{zed}
    child_{j} = first(args?) \mapsto fn! \\
    in!?_{path?_{j-1}} = merge((in?_{path_{j-1}}, fn!), first(args?))
    \where
    child_{j} \in in!?_{path?_{j-1}} \\
    child_{j} \not \in in?_{path?_{j-1}} \iff child_{j} \not= args? \\
    args? \in in?_{path?_{j-1}} \\
    args? \not \in in!?_{path?_{j-1}} \iff args? \not= child_{j}
  \end{zed}
\item retrace navigation back up from $in!?_{path?_{j-1}}$,
  updating the mapping at each $path?_{n} \in path?$ without touching any other mappings.
  \begin{zed}
    in!?_{path?_{j-1}} \ndres first(args?) = in?_{path?_{j-1}} \ndres first(args?) \iff args? \not= child_{j}
    \where
    args? \not= child_{j} \implies second(args?) \not= second(child_{j}) \\
    in!?_{path?_{j-1}} \ndres first(args?) \implies in!?_{path?_{j-1}} \dres ~(~\dom ~in!?_{path?_{j-1}} ~\setminus ~first(args?))
  \end{zed}
\item return $out!$ after the final update is made to $in?$.
  \begin{zed}
    child_{i} = atIndex(path?, i) \mapsto in!?_{path?_{i}} \\
    in!?_{path?_{i}} = merge((in?_{path?_{i}}, in!?_{path?_{i+1}}), atIndex(path?, i+1))
    \where
    out! = merge((in?, second(child_{i})), first(child_{i})) @ \\
    \t1 in? ~\ndres ~head(path?) ~=~ out! ~\ndres ~head(path?) \implies \\
    \t2 \forall (a,b) \in path? @ b = atIndex(path?, a) ~|~ \exists ~a @ in?_{a} = out!_{a} \iff a \not= head(path?)
  \end{zed}
\end{enumerate}

\subsection{Replace At, Append At and Update At}
In the summary of $walkBack$ above, the update at the target location within $in?$ takes place at step 4.
The result of step 4, $fn!$, will overwrite the mapping $args$ such that $fn!$ replaces $in?_{path?}$ due to $fn?_{nav} = merge~\_~$.
This results in the replacement of one mapping at each level of nesting such that the overall structure, composition and
size of $out!$ is comparable to $in?$ unless $fn?_{\delta}$ dictates otherwise. While the functionality of $fn_{nav}$ has been
constrained here to always be an overwriting process, the same constraint is not placed on $fn?_{\delta}$.
\subsubsection{Replace At}
The Primitive $replaceAt$ was first defined in terms of the Traversal Operations and then served as the starting point
for abstracting away aspects of functionality and delegating their responsibility to some passed in function until $WalkBack$
was reached. An alternate form of this formal definition is presented below such that $replaceAt$ is defined in terms of $walkBack$.
\begin{schema}{ReplaceAt[V, Collection, V]}
  WalkBack, Merge \\
  in?, with?, out!, fn!_{\delta} : V \\
  path? : Collection \\
  fn_{\delta} : V \pfun V \\
  replaceAt~\_ : V \cross Collection \cross V \bij V
  \where
  replaceAt = \langle walkBack~\_ \rangle \\ ~ \\
  out! = replaceAt(in?, path?, with?) = walkBack(in?, path?, fn_{\delta}, merge~\_) @ \\
  \t1 let \ ~ fn!_{\delta} == fn_{\delta}(getIn(in?, path?)) = with? \implies \\
  \t2 walkBack(in?, path?, fn_{\delta}, merge~\_) \equiv \\
  \t3 backProp(in?, path?, fn!_{\delta}, merge~\_) \equiv \\
  \t4 backProp(in?, path?, with?, merge~\_)
\end{schema}
\begin{itemize}
\item $fn_{\delta}$ is defined within $ReplaceAt$ as it performs a very simple task; ignore $getIn(in?, path?)$ and return $with?$
\item Here, $fn_{\delta}$ represents one of the main general categories of update; replacement of a value
  such that the result of the replacement is in no way dependent upon the thing being replaced.
\end{itemize}
The following examples were pulled from the section containing the first version of $ReplaceAt$ as they still hold true.
\begin{argue}
  replaceAt(X, \langle 2, foo, q \rangle, fn!) = \langle x_{0}, x_{1}, \ldata foo \mapsto \ldata bar \mapsto buz, x \mapsto y, q \mapsto ZZZ \rdata \rdata\rangle \\
  replaceAt(X, \langle 2, foo, x \rangle, fn!) = \langle x_{0}, x_{1}, \ldata foo \mapsto \ldata bar \mapsto buz, x \mapsto ZZZ \rdata \rdata\rangle \\
\end{argue}

\subsubsection{Append At}
In order to define the Primitive $appendAt$, the Traversal Operation $conj$ is used.
In order to demonstrate the usage of $conj$ as $fn?_{\delta}$ of $walkBack$,
a syntax not yet formally defined in this document is defined. It is an extension of the shorthand
$val_{index} = get(Val, index)$ as seen in examples like
\begin{argue}
  conj(x_{0}, false) = \langle true, false \rangle = \langle x_{0}, false \rangle\\
  conj(X, X) = \langle x_{0}, x_{1}, x_{2}, \langle x_{0}, x_{1}, x_{2} \rangle \rangle \\
\end{argue}
The following expands that usage to describe following some $path?$ into a $Collection$ or $KV$.
\begin{axdef}
  X_{path?} = getIn(X, path?)
  \where
  X_{\langle 1 \rangle} = x_{1} =  \langle a, b, c \rangle \\
  X_{\langle 1, 0 \rangle} = a
\end{axdef}
This syntax is used for the placeholder $X_{path?}$ so that the role of $fn?_{\delta}$ can be demonstrated within the arguments
passed to $walkBack$. This notation can be used to describe how arguments passed to a top level function get used within
component functions without writing the equivalent Z schema. This shorthand can also be used within Z schemas.
\begin{argue}
  walkBack(X, \langle 1 \rangle, map~\_(conj~\_, X_{\langle 1 \rangle}, a), merge~\_) = \langle x_{0}, \langle \langle a, a \rangle, \langle b, a \rangle, \langle c, a \rangle \rangle, x_{2} \rangle \\
  walkBack(X, \langle 1 \rangle, conj~\_(X_{\langle 1 \rangle}, a),  merge~\_) = \langle x_{0}, \langle a, b, c, a \rangle, x_{2} \rangle
\end{argue}
Additive updates are another common type of updating encountered when working with xAPI data. $Conj$ is a derivative
of $\cat$ but scoped to DAVE and used to define the Primitive $appendAt$.
\begin{schema}{AppendAt[V, Collection, V]}
  WalkBack, Conj, Merge \\
  in?, toEnd?, out! : V \\
  path? : Collection \\
  appendAt~\_ : V \cross Collection \cross V \bij V
  \where
  appendAt = \langle walkBack~\_ \rangle \\ ~ \\
  out! = appendAt(in?, path?, toEnd?) \equiv \\
  \t2 walkBack(in?, path?, conj~\_(in?_{path?}, toEnd?)~, merge~\_) \implies \\
  \t3 backProp(in?, path?, fn!_{\delta}, merge~\_) \iff \\
  \t4 fn!_{\delta} = fn?_{\delta}(in?_{path?}, toEnd?) @ \\
  \t5 fn?_{\delta}~\_(in?_{path?}, toEnd?) = fn?_{\delta} \rel (in?_{path?}, toEnd?) @ \\
  \t5 conj~\_(in?_{path?}, toEnd?) = conj~\_ \rel (in?_{path?}, toEnd?) \implies \\
  \t6 (fn?_{\delta} = conj~\_) ~\land \\
  \t6 (fn!_{\delta} \not= conj~\_(in?_{path?}, toEnd?)) ~\land~ \\
  \t6 (fn!_{\delta} = conj(in?_{path?}, toEnd?))
\end{schema}
This schema features a new notation which highlights evaluation nuances.
\begin{itemize}
\item $fn?_{\delta}$ is used to represent the function itself
\item $fn?_{\delta}~\_(in?_{path?}, toEnd?)$ is used to represent the relationship between the function and the arguments it WILL be passed
\item $fn!_{\delta} \equiv fn?_{\delta}(in?_{path?}, toEnd?)$ is used to represent the output of $fn?_{\delta}$ given the passed in arguments
\end{itemize}
Such that the following are all equivalent expressions.
\begin{zed}
  appendAt(in?, path?, toEnd?) \equiv \\
  \t1 walkBack(in?, path?, fn?_{\delta}, merge~\_) \equiv \\
  \t1 walkBack(in?, path?, conj~\_(in?_{path?}, toEnd?), merge~\_) \equiv \\
  \t1 walkBack(in?, path?, fn?_{\delta}~\_(in?_{path?}, toEnd?), merge~\_) \equiv \\
  \t2 backProp(in?, path?, fn!_{\delta}, merge~\_) \equiv \\
  \t2 backProp(in?, path?, conj(in?_{path?}, toEnd?), merge~\_)
\end{zed}
The following example demonstrates this usage.
\begin{axdef}
  walkBack(X, \langle 1 \rangle, map~\_(append~\_, X_{\langle 1 \rangle}, a), merge~\_) = \langle x_{0}, \langle \langle a, a \rangle, \langle b, a \rangle, \langle c, a \rangle \rangle, x_{2} \rangle \\
  \where
  map~\_(append~\_, X_{\langle 1 \rangle}, a) \equiv map~\_(append~\_(X_{\langle 1, n \rangle}, a) , X_{\langle 1 \rangle}, a) @ n \in \dom X_{\langle 1 \rangle}
\end{axdef}
The following examples demonstrate the functionality of $appendAt$.
\begin{argue}
  appendAt(X, \langle 1 \rangle, e) = \langle x_{0}, \langle a, b, c, e \rangle, x_{2} \rangle \\
  appendAt(X, \langle 2 \rangle, \langle 1, 2, 3 \rangle) = \langle x_{0}, x_{1}, \langle x_{2}, \langle 1, 2, 3 \rangle \rangle \rangle \\
  appendAt(X, \langle 0 \rangle, bar) = \langle \langle x_{0}, bar \rangle, x_{1}, x_{2} \rangle
\end{argue}

\subsubsection{Update At}
The Primitive $updateAt$ does not make any assumptions about how the relationship between
$getIn(in?, path?)$ and $fn!_{\delta}$ is established. This makes it possible to define both
$replaceAt$ and $appendAt$ using $updateAt$.

\begin{schema}{UpdateAt[V, Collection, (~\_~\pfun~\_)]}
  WalkBack, Merge \\
  in?, out! : V \\
  path? : Collection \\
  fn?_{\delta} : (~\_~\pfun~\_~) \\
  updateAt~\_ : V \cross Collection \cross (~\_~\pfun~\_) \bij V
  \where
  updateAt = \langle walkBack~\_ \rangle \\ ~ \\
  out! = updateAt(in?, path?, fn?_{\delta}) = \\
  \t2 walkBack(in?, path?, fn?_{\delta}, merge~\_) \implies \\
  \t3 backProp(in?, path?, fn!_{\delta}, merge~\_)
\end{schema}
\begin{itemize}
\item The item found at the target path $getIn(in?, path?)$ is passed to $fn?_{\delta}$ such that the calculation of the
  replacement $fn!_{\delta}$ CAN be dependent upon $getIn(in?, path?)$.
\end{itemize}
The following examples demonstrate the functionality of the Primitive $updateAt$
\begin{argue}
  updateAt(X, \langle 0 \rangle, array?~\_) = \langle false, x_{1}, x_{2} \rangle \\
  updateAt(X, \langle 1, 0 \rangle, fn?_{\delta}~\_(X_{\langle 1, 0 \rangle})) = \langle x_{0}, \langle z, b, c \rangle, x_{2} \rangle \iff fn?_{\delta}(X_{\langle 1, 0 \rangle}) = z
\end{argue}
and the following shows how $updateAt$ can be used to define $appendAt$
\begin{zed}
  appendAt(in?, path?, toEnd?) \equiv updateAt(in?, path?, conj~\_(in?_{path?}, toEnd?))
\end{zed}
and $replaceAt$.
\begin{zed}
  replaceAt(in?, path?, with?) \equiv updateAt(in?, path?, merge~\_((in?_{path?_{j-1}}, with?), \langle ~path?_{j} \rangle~))
\end{zed}
\end{document}
