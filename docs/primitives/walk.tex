\documentclass[../main.tex]{subfiles}

\begin{document}
There will be many Primitives used within Algorithm definitions in DAVE but
navigation into a nested $Collection$ or $KV$ is most likely to be used
across nearly all Algorithm definitions. In the following section,
helper Operations are introduced for navigation into and back out of a
nested Value. These Operations are then used to define the common Primitives centered around
traversal of nested data structures ie. xAPI Statements and Algorithm State.

\subsection{Traversal Operations}
\begin{schema}{Get[V, Collection]}
  in?, v! : V \\
  id? : Collection \\
  get~\_ : V \cross Collection \surj V
  \where
  v! = get(in?, id?) @\\
  \t1 \ ~  = (atIndex(in?, head(id?)) \iff (array?(in?) = true) ~\land~ (head(id?) \in \nat)) ~\lor \\
  \t1 \ \ \ \ ~ (atKey(in?, head(id?)) \iff (array?(in?) = false) ~ \land ~ (map?(in?) = true))
\end{schema}
\begin{itemize}
  \item Navigation down into either a $Collection$ or $KV$ based on the type of $in?$
\end{itemize}
\begin{schema}{Merge[(V, V), Collection]}
  parent?, child?, parent! : V \\
  at? : Collection \\
  merge~\_ : (V \cross V) \cross Collection \bij V
  \where
  parent! = merge((parent?, child?), at?) @ \\
  \t1 \ \ \ \ ~ = (associate(parent?, head(at?), child?) \\
  \t3 \iff map?(parent?) = true) ~ \lor \\
  \t2 \ ~ (update(parent?, child?, head(at?)) \\
  \t3 \iff (array?(parent?) = true) ~\land~ (head(at?) \in \nat))
\end{schema}
\begin{itemize}
  \item Updating of $parent?$ to include $child?$ at location indicated by $head(at?)$
\end{itemize}
\subsection{Traversal Primitives}
The helper Operations defined above are used to describe the traversal of a heterogeneous nested Value.
In the following subsections, examples which demonstrate the functionality of Primitives will be passed $X$ as $in?$.
\begin{argue}
  X = \langle x_{0}, x_{1}, x_{2} \rangle \\
  \t1 x_{0} = true \\
  \t1 x_{1} = \langle a, b, c \rangle \\
  \t1 x_{2} = \ldata foo \mapsto \ldata bar \mapsto buz, x \mapsto y , z \mapsto \langle 3, 2, 1 \rangle \rdata \rdata \\
  fn! = fn(X_{path?_{j-1}}, arg?) = ZZZ & $fn~\_$ always returns $ZZZ$
\end{argue}

\subsubsection{Get In}
Collection and KV have different Fundamental Operations for navigation into, value extraction from
and application of updates to. Navigation into an arbitrary Value without concern
for its type is a useful tool to have and has been defined as the Primitive $getIn$.
\begin{schema}{GetIn[V, Collection]}
  Get, Recur \\
  in?, atPath! : V \\
  path? : Collection \\
  getIn~\_ : V \cross Collection \surj V
  \where
  getIn = \langle get~\_~, recur~\_ \rangle \bsup \#~path? - 1 \esup \\ ~ \\
  atPath! = getIn(in?, path?) @ \\
  \ \forall n : i~..~j-1 @ j = first(last(path?)) \implies first(j, path?_{j}) ~|~ \exists ~ down_{n} @\\
  \t1 let \ \ path?_{n} == tail(path?)\bsup n-i \esup \\
  \t1 \ \ \ \ \ down_{i} == get(in?, path?_{n}) \implies \\
  \t5 atIndex(in?, head(path?)) ~\lor \\
  \t5 atKey(in?, head(path?)) \iff n = i\\
  \t1 \ \ \ \ \ down_{n} == recur(down_{i}, path?_{n}, get~\_)\bsup j - 1 \esup \\
  \t1 \ \ \ \ \ down_{j-1} == get(down_{n}, path?_{n}) \iff n = j - 2 \\ ~ \\
  atPath! = down_{j} = get(down_{j-1}, path?_{n}) @ \\
  \t5 \ path?_{n} \equiv (path? \extract j) \implies \\
  \t6 \ \langle j \mapsto atIndex(path?, j) \rangle \iff n = j-1 \\
\end{schema}
The following examples demonstrate the functionality of the Primitive $getIn$
\begin{argue}
  getIn(X, \langle 1, 1 \rangle) = b \\
  getIn(X, \langle 0 \rangle) = true \\
  getIn(X, \langle 2, foo, z, 0 \rangle) = 3
\end{argue}
% FIXME: wordsmith...sounds like english translation of Z schema. Can be said in a more clear/concise way.
Additionally, the propagation of an update, starting at some depth within a passed in Value and bubbling up to the top level,
such that the update is only applied to values along a specified path as necessary, is also a useful tool to have.
The following sections introduce Primitives which address performing these types of updates and ends with a summary of
the functional steps described in the sections bellow. $replaceAt$ is introduced first and serves as a point of comparison
when describing the more abstract Primitives $backProp$ and $walkBack$.

\subsubsection{Replace At}
The schema $ReplaceAt$ uses the helper Operation $merge$ to apply updates while climbing up from some arbitrary depth.
\begin{schema}{ReplaceAt[V, Collection, V]}
  GetIn, Merge \\
  in?, with?, out! : V \\
  path? : Collection \\
  replaceAt~\_ : V \cross Collection \cross V \bij V
  \where
  replaceAt = \langle \langle getIn~\_~, ~merge~\_~ \rangle , ~recur~\_ \rangle \bsup \#~path? - 1 \esup \\ ~ \\
  out! = replaceAt(in?, path?, with?) @ \\
  \t1 \forall n : i~..~j-1 @ (i = first(head(path?))) ~\land~ (j = first(last(path?))) ~|~ \exists ~ parent_{n} @ \\
  \t3 let \ ~ path?_{n} == tail(path?)\bsup n-i \esup \\
  \t2 parent_{n} = recur(parent_{n-1}, path?_{n}, get~\_)\bsup j - 1 \esup \implies \\
  \t3 let \ ~ parent_{i} == getIn(in?, path?_{n}) \iff n = i \\
  \t3 \ \ \ \ ~ parent_{i+1} == getIn(parent_{i}, path?_{n}) \iff n = i+1 \\
  \t3 \ \ \ \ ~ parent_{j-1} == getIn(parent_{j-2}, path?_{n}) \iff n = j-1 \\
  \t3 parent_{j} =  getIn(parent_{j-1}, (path? \extract j))
  \\ ~ \\
  \t1 \forall z : p~..~q @ (p = j-1) ~\land ~ (q = i + 1) \implies \\
  \t4 ((z = p \iff n = j-1) \land (z = q \iff n = i + 1)) ~|~ \exists ~ child_{z} @ \\
  \t3 let \ \ path?_{rev} == rev(path?) \\
  \t3 \ \ \ \ \ path?_{z} == tail(path?_{rev})\bsup p-z+1 \esup \\
  \t2 child_{z} = recur((parent_{n}, child_{n+1}), path?_{z}, merge~\_ ) \\
  \t3 let \ ~ child_{p} == merge((parent_{n}, with?), path?_{z}) \iff z = p \implies n = j-1 \\
  \t3 \ \ \ \ ~ child_{p+1} == merge((parent_{n}, child_{p}), path?_{z}) \iff n = j - 2 ~\land ~p = j - 1 \\
  \t3 \ \ \ \ ~ child_{q} == merge((parent_{n}, child_{q+1}), path?_{z}) \iff z = q \implies n = i + 1
  \\~\\
  out! = merge((in?, child_{q}), path?_{n}) \equiv merge((in?, child_{q}), (path? \extract i)) \iff (n = i = q - 1)
\end{schema}
\begin{itemize}
\item The range of indices $i~..~j-1$ is used to describe navigation into some Value given $path?$
  \begin{itemize}
  \item Used to reference preceding level of depth
  \item keeps track of parent from previous steps
  \end{itemize}
\item The range of indices $p~..~q$ is used to describe navigation up from target depth indicated by $path?$
  \begin{itemize}
  \item Used to reference current level of depth
  \item keeps track of child after the update has been applied
  \end{itemize}
\item The propagation of the update starts with $child_{p}$
  \begin{itemize}
  \item $with?$ is added to $parent_{j-1}$ at $get~(path?, \langle ~j~ \rangle)$
  \item parent nodes need to be notified of the change within their children
  \end{itemize}
\end{itemize}
The following examples demonstrate the functionality of the Primitive $replaceAt$
\begin{argue}
  replaceAt(X, \langle 2, foo, q \rangle, fn!) = \langle x_{0}, x_{1}, \ldata foo \mapsto \ldata bar \mapsto buz, x \mapsto y, q \mapsto ZZZ \rdata \rdata\rangle \\
  replaceAt(X, \langle 2, foo, x \rangle, fn!) = \langle x_{0}, x_{1}, \ldata foo \mapsto \ldata bar \mapsto buz, x \mapsto ZZZ \rdata \rdata\rangle \\
\end{argue}
This Primitive can be made more general purpose by replacing $merge$ with a placeholder $fn?$
representing a passed in Operation or Primitive.

\subsubsection{Back Prop}
Being able to pass a function as an argument allows for, in this context, the arbitrary handling of
how update(s) are applied at each level of nesting. The arbitrary $fn?$ should expect
a (Parent, Child) tuple and a Collection of indices as arguments and return a potentially modified version of the parent.
\begin{schema}{BackProp~[V, Collection, V, (~\_\pfun~\_~)]}
  GetIn \\
  in?, fnSeed?, out! : V \\
  path? : Collection \\
  fn? : (~\_\pfun~\_~) \\
  backProp~\_ : V \cross Collection \cross V \cross (~\_\pfun~\_~) \bij V
  \where
  backProp = \langle \langle getIn~\_~, fn?~\_ \rangle , ~recur~\_ \rangle \bsup \#~path? - 1 \esup \\ ~ \\
  out! = backProp~(in?, path?, fnSeed?, fn?) @ \\
  \t1 \forall n : i~..~j-1 @ (i = first(head(path?))) ~\land~ (j = first(last(path?))) ~|~ \exists ~ parent_{n} @ \\
  \t3 let \ ~ path?_{n} == tail(path?)\bsup n-i \esup \\
  \t2 parent_{n} = recur(parent_{n-1}, path?_{n}, get~\_)\bsup j - 1 \esup \implies \\
  \t3 let \ ~ parent_{i} == getIn(in?, path?_{n}) \iff n = i \\
  \t3 \ \ \ \ ~ parent_{i+1} == getIn(parent_{i}, path?_{n}) \iff n = i+1 \\
  \t3 \ \ \ \ ~ parent_{j-1} == getIn(parent_{j-2}, path?_{n}) \iff n = j-1 \\
  \t3 parent_{j} =  getIn(parent_{j-1}, (path? \extract j))
  \\ ~ \\
  \t1 \forall z : p~..~q @ (p = j-1) ~\land ~ (q = i + 1) \implies \\
  \t4 ((z = p \iff n = j-1) \land (z = q \iff n = i + 1)) ~|~ \exists ~ child_{z} @ \\
  \t3 let \ \ path?_{rev} == rev(path?) \\
  \t3 \ \ \ \ \ path?_{z} == tail(path?_{rev})\bsup p-z+1 \esup \\
  \t2 child_{z} = recur((parent_{n}, child_{n+1}), path?_{z}, fn?) \\
  \t3 let \ ~ child_{p} == fn?((parent_{n}, fnSeed?), path?_{z}) \iff z = p \implies n = j-1 \\
  \t3 \ \ \ \ ~ child_{p+1} == fn?((parent_{n}, child_{p}), path?_{z}) \iff n = j - 2 ~\land ~p = j - 1 \\
  \t3 \ \ \ \ ~ child_{q} == fn?((parent_{n}, child_{q+1}), path?_{z}) \iff z = q \implies n = i + 1
  \\~\\
  out! = fn?((in?, child_{q}), path?_{n}) \equiv fn?((in?, child_{q}), (path? \extract i)) \iff (n = i = q - 1)
\end{schema}
The schema $ReplaceAt$ was introduced before $BackProp$ so the process underlying both
could be explicitly demonstrated and defined. The hope is that this made the introduction of
the more abstract Primtive $backProp$ easier to follow. A quick comparison of $ReplaceAt$ and
$BackProp$ reveals that the only major difference between them is $fn?$ vs $merge~\_$.
This implies the Primitive $backProp$ can be used to replicate $replaceAt$.
\begin{zed}
  replaceAt(in?, path?, with?) \equiv \\
  \t1 backProp~(in?, path?, fnSeed?, merge~\_) \iff with? = fnSeed?
\end{zed}
Above highlights the arguments $with? ~\land~ fnSeed?$ which serve the same purpose within $backProp$ and $replaceAt$.
\begin{itemize}
\item Within $ReplaceAt$, the naming $with?$ indicates its usage with respect to $merge$ and the overall functionality of the Primitive
\item Within $BackProp$, the naming $fnSeed?$ indicates that the usage of the variable within $fn?$ is unknowable but this value
will be passed to $fn?$ on the very first iteration of the Primitive
\end{itemize}
In both cases, the variable is put into a tuple and passed to $fn?$.
\begin{zed}
  backProp(X, \langle 2, foo, x \rangle, fn!, merge~\_) = \langle x_{0}, x_{1}, \ldata foo \mapsto \ldata bar \mapsto buz, x \mapsto ZZZ \rdata \rdata\rangle \\
\end{zed}
The notable limitation of $backProp$ are enumerated in the bullets bellow and the Primitive $walkBack$ is introduced to address them.
\begin{itemize}
\item expectation of a seeding value ($fnSeed?$) as a passed in argument
\item the general dismissal of the value ($parent_{j}$) located at $path?$ which is potentially being overwritten
\end{itemize}

\subsubsection{Walk Back}

In the Primitive $walkBack$, $fnSeed?$ is assumed to be the result of a function $fn?_{\delta}$
which is passed in as an argument. $fn?_{\delta}$ will be passed $parent_{j}$ as an argument in order to produce $fnSeed?$.
This Value will then be used exactly as it was in $backProp$ given $walkBack$ expects another function argument $fn?_{nav}$.
\begin{zed}
  walkBack(in?, path?, fn?_{\delta}, fn?_{nav})
\end{zed}
In fact, the usage of $fn?_{nav}$ in $WalkBack$ is exactly the same as the usage of $fn?$
in $BackProp$ as $fn?_{nav}$ is passed to $backProp$ as $fn?$.

\begin{schema}{WalkBack[V, Collection, (~\_~\pfun~\_~), (~\_~\pfun~\_~)]}
  BackProp \\
  in?, out! : V \\
  path? : Collection \\
  fn?_{\delta}, fn?_{nav} : (~\_~\pfun~\_~) \\
  walkBack~\_ : V \cross Collection \cross (~\_~\pfun~\_~) \cross (~\_~\pfun~\_~) \bij V
  \where
  walkBack = \langle getIn~\_ ~, ~fn?_{\delta}~\_~, ~backProp~\_ \rangle \\ ~ \\
  out! = walkBack(in?, path?, fn?_{\delta}, fn?_{nav}) @ \\
  \t1 let \ ~ fnSeed == fn?_{\delta}(getIn(in?, path?)) \\
  \t1 = backProp(in?, path?, fnSeed, fn?_{nav})
\end{schema}
By replacing $fnSeed?$ with $fn?_{\delta}$ as an argument
\begin{itemize}
\item $walkBack$ can be used to describe predicate based traversal of $in?$
\item $walkBack$ can be used to update Values at arbitrary nesting within $in?$ and at the same time describe how those changes affect the rest of $in?$
\end{itemize}
$walkBack$ serves as a graph traversal template Primitive whose behavior is defined in terms of the nodes within $in?$
and the interpertation of those nodes via $fn?_{\delta}$ and $fn?_{nav}$. This establishes the means for defining Primitives which can make
longitudinal updates as needed before making horizonal movements through some $in?$. In order for $backProp$ to be used in the same way, the required state must be managed by
\begin{itemize}
\item $fn_{nav}$
\item some higher level Primitive that contains $backProp$ (see $WalkBack$)
\end{itemize}
This important difference means $walkBack$ can be used to replicate $backProp$ but the opposite is not always true.
\begin{zed}
  walkBack(in?, path?, fn?_{\delta}, fn?_{nav}) \equiv \\
  \t1 backProp(in?, path?, fnSeed?, fn?_{nav}) \iff fnSeed? = fn?_{\delta}(getIn(in?, path?))
\end{zed}
This means $replaceAt$ can also be replicated.
\begin{zed}
  replaceAt(in?, path?, with?) \equiv \\
  \t1 (backProp~(in?, path?, fnSeed?, merge~\_) \iff with? = fnSeed?) \equiv \\
  \t2 walkBack(in?, path?, fn?_{\delta}, merge~\_) \iff \\
  \t3 fn?_{\delta}(getIn(in?, path?)) = fnSeed? = with?
\end{zed}
The following examples demonstrate the functionality of $walkBack$
\begin{argue}
  walkBack(X, \langle 0 \rangle, array?~\_~, merge~\_) = \langle false, x_{1}, x_{2} \rangle \\
  walkBack(X, \langle 2, qux \rangle, fn~\_~, merge~\_) = \langle x_{0}, x_{1}, (x_{2} ~\cup~ qux \mapsto ZZZ)\rangle \\
  walkBack(X, \langle 1 \rangle, map(succ~\_, x_{1}, 1), merge~\_) = \langle x_{0}, \langle b, c, d \rangle, x_{2} \rangle \\
  walkBack(X, \langle 1, 0 \rangle, succ~\_, merge~\_) = \langle x_{0}, \langle b, b, c \rangle, x_{2} \rangle
\end{argue}

\subsection{Summary}
The following is a summary of the general process which has been described in the previous sections.
The variable names here are NOT intended to be 1:1 with those in the formal definitions (but
there is some overlap) and the summary utilizes the Traversal Operations defined at the start of the section.
\begin{enumerate}
\item navigate down into the provided value $in?$ up until the second to last value $in?_{path?_{j-1}}$ as described by the provided $path?$
  \begin{zed}
    in?_{path?_{j-1}} : V
    \where
    path?_{j-1} \implies path? ~\ndres~ j \implies path? \dres ~(~\dom ~path? ~\setminus ~\{j\})
  \end{zed}
\item extract any existing data mapped to $atIndex(path?, j)$ from the result of step 1
  \begin{zed}
    in?_{path?} : V
    \where
    path? \implies path?_{j-1} ~\cup~ (j, atIndex(path?, j))
  \end{zed}
\item create the mapping $(atIndex(path?, j), in?_{path?})$ labeled here as $args?$
  \begin{zed}
    args? = (atIndex(path?, j), in?_{path?})
    \where
    args? \in in?_{path?_{j-1}}\\
    first(args?) = atIndex(path?, j)
  \end{zed}
\item pass $in?_{path?}$ to the provided function $fn?$ to produce some output $fn!$
  \begin{zed}
    fn! = fn?(second(args?)) = fn?(in?_{path?})
   \end{zed}
\item replace the previous mapping $args?$ within $in?_{path?_{j-1}}$ with $fn!$ at $atIndex(path?, j)$
  \begin{zed}
    child_{j} = first(args?) \mapsto fn! \\
    in!?_{path?_{j-1}} = merge((in?_{path_{j-1}}, fn!), first(args?))
    \where
    child_{j} \in in!?_{path?_{j-1}} \\
    child_{j} \not \in in?_{path?_{j-1}} \iff child_{j} \not= args? \\
    args? \in in?_{path?_{j-1}} \\
    args? \not \in in!?_{path?_{j-1}} \iff args? \not= child_{j}
  \end{zed}
\item retrace navigation back up from $in!?_{path?_{j-1}}$,
  updating the mapping at each $path?_{n} \in path?$ without touching any other mappings.
  \begin{zed}
    in!?_{path?_{j-1}} \ndres first(args?) = in?_{path?_{j-1}} \ndres first(args?) \iff args? \not= child_{j}
    \where
    args? \not= child_{j} \implies second(args?) \not= second(child_{j}) \\
    in!?_{path?_{j-1}} \ndres first(args?) \implies in!?_{path?_{j-1}} \dres ~(~\dom ~in!?_{path?_{j-1}} ~\setminus ~first(args?))
  \end{zed}
\item return $out!$ after the final update is made to $in?$.
  \begin{zed}
    child_{i} = atIndex(path?, i) \mapsto in!?_{path?_{i}} \\
    in!?_{path?_{i}} = merge((in?_{path?_{i}}, in!?_{path?_{i+1}}), atIndex(path?, i+1))
    \where
    out! = merge((in?, second(child_{i})), first(child_{i})) @ \\
    \t1 in? ~\ndres ~head(path?) ~=~ out! ~\ndres ~head(path?) \implies \\
    \t2 \forall (a,b) \in path? @ b = atIndex(path?, a) ~|~ \exists ~a @ in?_{a} = out!_{a} \iff a \not= head(path?)
  \end{zed}
\end{enumerate}
\end{document}
