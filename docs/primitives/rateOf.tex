\documentclass[../main.tex]{subfiles}
\begin{document}

\subsection{Rate Of}
Calculates the number of times something occured within an interval of time given a unit of time.
$$rateOf(nOccurances, start, end, unit)$$
Where the output translates to: the rate of occurance per unit within interval

\subsubsection{Arguments}
\begin{itemize}
\item $nOccurances$ is the number of times something happened and should be an Integer
\item $start$ is an ISO 8601 timestamp which serves as the first timestamp within the interval
\item $end$ is an ISO 8601 timestamp which servers as the last timestamp within the interval
\item $unit$ is a String Enum representing the unit of time
\end{itemize}

\subsubsection{Relevant Operations}
\begin{itemize}
\item isoToUnixEpoch
\item $timeunit \to seconds$
\end{itemize}

\subsubsection{Summary}
$rateOf$ determines the number of seconds within the interval $start \to end$
$$intervalSeconds = isoToUnixEpoch(end) - isoToUnixEpoch(start)$$
and resolves the numer of seconds corresponding to $unit$
$$unitSeconds = timeunit \to seconds(unit)$$
so that the interval can be converted from Seconds to $unit$
$$per = intervalSeconds / unitSeconds$$
and the rate can be calculated
$$rateOf(nOccurances, start, end, unit) \equiv nOccurances \div per$$

\subsubsection{Usage of Operations}
The Operations used within $rateOf$ convert from a String to a Integer
\begin{itemize}
\item $isoToUnixEpoch$ is used to convert
  $$end \ \land start \to \real$$
\item $timeunit \to seconds$ is used to convert
  $$unit \to \real$$
\end{itemize}
The only other functionality required by $rateOf$ is supplied via basic arithmetic

\subsubsection{Example output}
Given an example $start$ and $end$
$$start = 2015-11-18T12:17:00Z$$
$$end = 2015-11-18T14:17:00Z$$
Then the Unix Epoch of each is
$$nStart = 1447849020$$
$$nEnd = 1447856220$$
Which provides an interval range (in seconds)
$$intervalSeconds = nEnd - nStart = 7200$$
which is divided by $timeunit \to seconds(unit)$ to derive per $unit$
$$per = 7200 \iff unit = second \implies 7200 / 1$$
$$per = 120 \iff unit = minute \implies 7200 / 60$$
$$per = 2 \iff unit = hour \implies 7200 / 3600$$
such that if
$$nOccurances = 10$$
and $unit = second$ then the output is 0.001389 occurances per second within $start \to end$
$$rateOf(nOccurances, start, end, second) \equiv 10 / 7200 \equiv 0.001389$$
and $unit = minute$ then the output is 0.0833 occurances per minute within $start \to end$
$$rateOf(nOccurances, start, end, minute) \equiv 10 / 120 \equiv 0.0833$$
and $unit = hour$ then the output is 5 occurances per hour within $start \to end$
$$rateOf(nOccurances, start, end, hour) \equiv 10 / 2 \equiv 5$$
\end{document}
