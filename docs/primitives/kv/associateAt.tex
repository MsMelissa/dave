\documentclass[../../main.tex]{subfiles}
\begin{document}
\subsubsection{Associate At}
The Primitive $associateAt$ establishes a relationship between $k?_{j}$ and $v?$
at the nesting $k?_{i}~..k?_{j-1}$ within $m!$
\begin{axdef}
  k? = \langle k?_{i}~..~k?_{j} \rangle \\
  (k?_{j}, m?_{k?}) \in m? ~\lor ~(k?_{j}, m?_{k?}) \not \in m? \\
  (k?_{j}, m?_{k?}) \not \in m! \iff m?_{k?} \not = v?\\
  (k?_{j}, v?) \in m!
  \where
  m! = associateAt(m?, k?, v?)
\end{axdef}
This implies that any existing mapping at $k?_{j} \in m?$ will be overwritten by $associateAt$
but an existing mapping is not a precondition.
The following helper Operation $getFirstKey$ is introduced
to establish navigation into a nested Map given a Collection of Keys.
\begin{schema}{GetFirstKey[KV, Collection]}
  m? : KV \\
  k? : Collection \\
  v! : V \\
  getFirstKey~\_ : KV \cross Collection \surj V
  \where
  v! = getFirstKey(m?, k?) @ v! = atKey(m?, head(k?))
\end{schema}
This allows for the navigation into a nested Map to be defined as $\langle getFirstKey~\_~, recur~\_ \rangle\bsup \#~k?-1 \esup$
which represents a step down for each $k \in (k? \setminus ~k?_{j})$. Once at $k?_{j-1}$, the mapped value $v_{j-1}$ has $(k?_{j}, v?)$ added to it.
This update is localized within $m?$ and all other mappings within $m?$ are left alone.
\begin{schema}{AssociateAt[KV, Collection, V]}
  GetFirstKey, Recur \\
  m?, m! : KV \\
  k? : Collection @ \forall k?_{n} \in k?_{\langle i~..~j \rangle} ~|~ k_{n} : K \\
  v? : V \\
  associateAt~\_ : KV \cross Collection \cross V \bij KV\
  \where
  associateAt = \langle \langle atFirstKey~\_~, recur~\_~ \rangle\bsup \#~k?-1 \esup, \langle associate~\_ \rangle \bsup \#~k \esup ~\rangle \\
  m! = associateAt(m?, k?, v?) @ \\
  \t1 \forall n : i~..~j-1 \in \dom k? @ j = first(last(k?)) \implies first(j, k?_{j}) ~|~ \exists_1 v_{n} @\\
  \t2 let~ ~ ~ \ c_{n} == tail(k?)\bsup n - i \esup \\
  \t3 v_{i} == getFirstKey(m?, ~c_{n}) \implies \\
  \t4 c_{n} = k? \iff n = i @ v_{i} = atKey(m?, head(k?)) \\
  \t3 v_{n} = recur(v_{i}, c_{n}, getFirstKey~\_)\bsup j-1 \esup \\
  \t3 v_{j-1} == getFirstKey(v_{n}, (j-1 \extract k?)) \iff n = j-2 \\
  \t3 v_{j} == associate(v_{j-1}, last(k?), v?) \implies \ldata k?_{j} \mapsto v? \rdata \cup v_{j-1} \ndres k?_{j}\\
  \ \ \ ~~ = (v_{j} \cup v_{n - succ(1)} \ndres k?_{n - succ(0)})\bsup j-1 \esup @ n \leq j-1 \implies \\
  \t2 v_{j-4} \cup (v_{j-3} \cup (v_{j} \cup v_{j-2} \ndres k?_{j-1}) \ndres k?_{j-2}) \ndres k?_{j-3} @ \\
  m! = associate(m?, k?_{i}, associate(v_{i}, k?_{n}, associate(v_{n}, k?_{j-1}, associate(v_{j-1}, k?_{j}, v?))))
\end{schema}
In the schema above, the localization of the change and the retention of the other mappings is indicated via
\begin{zed}
  (v_{j} \cup v_{n - succ(1)} \ndres k?_{n - succ(0)})\bsup j-1 \esup @ n \leq j-1 \implies \\
  \t2 v_{j-4} \cup (v_{j-3} \cup (v_{j} \cup v_{j-2} \ndres k?_{j-1}) \ndres k?_{j-2}) \ndres k?_{j-3}
\end{zed}
and is the reason why $\langle associate~\_ \rangle \bsup \#~k \esup$ is included in the definition of $associateAt$
and is equivalent to
\begin{zed}
  associate(m?, k?_{i}, associate(v_{i}, k?_{n}, associate(v_{n}, k?_{j-1}, associate(v_{j-1}, k?_{j}, v?))))
\end{zed}
which gracefully walks into and  back out of a $KV$ regardless of $k?_{n} \in m?_{k?_{n-1}}$.
\begin{argue}
  M = \ldata k_{i} \mapsto v_{i}, \ k_{n} \mapsto \ldata k_{ni} \mapsto v_{ni}, \ k_{nj} \mapsto v_{nj} \rdata \rdata \\
  \t1 associateAt(M, \langle k_{n}, k_{nn} \rangle, v?) =
  \ldata k_{i} \mapsto v_{i}, \ k_{n} \mapsto \ldata k_{ni} \mapsto v_{ni}, \ k_{nj} \mapsto v_{nj}, \ k_{nn} \mapsto v? \rdata \rdata \\
  \t1 associateAt(M, \langle k_{j}, k_{ji} \rangle, v?) =
  \ldata k_{i} \mapsto v_{i}, \ k_{n} \mapsto \ldata k_{ni} \mapsto v_{ni}, \ k_{nj} \mapsto v_{nj} \rdata, k_{j} \mapsto \ldata k_{ji} \mapsto v? \rdata \rdata \\
  \t1 associateAt(M, \langle k_{i} \rangle, v?) =
  \ldata k_{i} \mapsto v?, \ k_{n} \mapsto \ldata k_{ni} \mapsto v_{ni}, \ k_{nj} \mapsto v_{nj} \rdata \rdata \\
  \t1 associateAt(M, \langle k_{i}, k_{ii} \rangle, v?) =
  \ldata k_{i} \mapsto \ldata k_{ii} \mapsto v? \rdata, \ k_{n} \mapsto \ldata k_{ni} \mapsto v_{ni}, \ k_{nj} \mapsto v_{nj} \rdata \rdata \\
\end{argue}
The last example demonstrates what happens when the value at some key is not a Map.
\begin{zed}
  atKey(v_{i}, k_{ii}) = \emptyset \iff k_{ii} \not \in \dom v_{i} @ \\
  \t1 associateAt(M, \langle k_{i}, k_{ii} \rangle, v?) \implies \\
  \t2 associate(M, k_{i}, associate(\emptyset, k_{ii}, v?))
\end{zed}
\end{document}
