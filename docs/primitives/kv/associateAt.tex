\documentclass[../../main.tex]{subfiles}
\begin{document}
\subsubsection{Associate At}
The Primitive $associateAt$ establishes a relationship between $k?_{j}$ and $v?$
at the nesting $k?_{i}~..k?_{j-1}$ within $m!$
\begin{axdef}
  k? = \langle k?_{i}~..~k?_{j} \rangle \\
  (k?_{j}, m?_{k?}) \in m? ~\lor ~(k?_{j}, m?_{k?}) \not \in m? \\
  (k?_{j}, m?_{k?}) \not \in m! \iff m?_{k?} \not = v?\\
  (k?_{j}, v?) \in m!
  \where
  m! = associateAt(m?, k?, v?)
\end{axdef}
This implies that any existing mapping at $k?_{j} \in m?$ will be overwritten by $associateAt$
but an existing mapping is not a precondition. The following helper Operation $getFirstKey$ is introduced
to help with the recursive nature of $associateAt$.
\begin{schema}{GetFirstKey[KV, Collection]}
  m? : KV \\
  k? : Collection \\
  v! : V \\
  getFirstKey~\_ : KV \cross Collection \surj V
  \where
  v! = getFirstKey(m?, k?) @ v! = atKey(m?, head(k?))
\end{schema}
This allows for the recursive aspect of $associateAt$ to be defined as $\langle getFirstKey~\_~, recur~\_ \rangle\bsup \#~k?-1 \esup$
\begin{schema}{AssociateAt[KV, Collection, V]}
  GetFirstKey, Recur \\
  m?, m! : KV \\
  k? : Collection @ \forall k?_{n} \in k?_{\langle i~..~j \rangle} ~|~ k_{n} : K \\
  v? : V \\
  associateAt~\_ : KV \cross Collection \cross V \bij KV\
  \where
  associateAt = \langle \langle atFirstKey~\ recur~\_~ \rangle\bsup \#~k?-1 \esup, associate~\_  \rangle \\
  m! = associateAt(m?, k?, v?) @ \\
  \t1 \forall n : i~..~j-1 \in k? @ j = first(last(k?)) \implies first(j, k?_{j}) ~|~ \exists_1 v_{n} @\\
  \t2 let~ ~ ~ \ c_{n} == tail(k?)\bsup n - i \esup \\
  \t3 v_{i} == getFirstKey(m?, ~c_{n}) \implies \\
  \t4 c_{n} = k? \iff n = i @ v_{i} = atKey(m?, head(k?)) \\
  \t3 v_{n} = recur(v_{i}, c_{n}, getFirstKey~\_)\bsup j-1 \esup \\
  \t3 v_{j-1} == getFirstKey(v_{n}, (j-1 \extract k?)) \iff n = j-2 \\
  \t3 v_{j} == associate(v_{j-1}, last(k?), v?) \implies \ldata k?_{j} \mapsto v? \rdata \cup v_{j-1} \ndres k?_{j}\\
  \ \ \ ~~ = (v_{j} \cup v_{n - succ(1)} \ndres k?_{n - succ(0)})\bsup j-1 \esup @ n \leq j-1 \implies \\
  \t2 v_{j-4} \cup (v_{j-3} \cup (v_{j} \cup v_{j-2} \ndres k?_{j-1}) \ndres k?_{j-2}) \ndres k?_{j-3}
\end{schema}
No other mappings which exist at the various levels of depth will be altered
% WIP - text + demonstration of properties
\end{document}
