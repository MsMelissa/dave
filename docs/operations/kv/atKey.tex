\documentclass[../../main.tex]{subfiles}
\begin{document}
\subsubsection{At Key}
The operation $atKey$ will return the Value $v$ at some specified Key $k$.
\begin{schema}{AtKey[KV, K]}
  m? : KV \\
  v! : V \\
  k? : K \\
  atKey~\_ : KV \cross K \surj V
  \where
  v! = atKey(m?, k?) @ \\
  \t2 let ~~ coll == ((\seq m?) \filter (k?, m?_{k?})) \implies \langle (k?, m?_{k?}) \rangle \iff k? \in \dom m? \\
  \ \ \ = (second(head(coll)) \iff k? \mapsto m?_{k?} \in coll) ~\lor \\
  \t1 (\emptyset \iff k? \not \in \dom m?)
\end{schema}
In the schema above, $coll$ is the result of filtering for $(k?, m?_{k?})$ within $\seq m?$.
If the mapping was in the original $m?$, it will also be in the sequence of mappings. This means
we can filter over the sequence to look for the mapping and if found, it is returned as $\langle (k?, m?_{k?}) \rangle$.
To return the mapping itself, $head(coll)$ is used to extract the mapping such that the value mapped to $k?$ can be returned.
\begin{zed}
  v! = atKey(m?, k?) = second(head(coll)) = m?_{k?} @ m?_{k?} : V \iff k? \in \dom m?
\end{zed}
The follow examples demonstrate the properties of $atKey$
\begin{argue}
  M = \ldata k_{0}v_{k_{0}}, k_{1}v_{k_{1}} \rdata \\
  \t1 k_{0} = abc \ \land  v_{k_{0}} = 123 & $k_{0}v_{k_{0}} = abc \mapsto 123$ \\
  \t1 k_{1} = def \ \land v_{k_{1}} = xyz \mapsto 456 & $k_{1}v_{k_{1}} = def \mapsto xyz \mapsto 456$ \\
  atKey(M, abc) = 123 \\
  atKey(M, def) = xyz \mapsto 456 \\
  atKey(M, foo) = \emptyset
\end{argue}
\end{document}
