\documentclass[../../main.tex]{subfiles}
\begin{document}
\subsubsection{At Index}
The operation $atIndex$ will return the Value at a
specified numeric index within a Collection or
an empty Collection if there is no value at the specified index.
\begin{schema}{AtIndex[Collection, \nat]}
  idx? : \nat\\
  coll? : Collection \\
  atIndex~\_ : Collection \cross \nat \surj V
  \where
  \# ~ idx? = 1 \\
  coll! = atIndex(coll?, idx?) = (head ~(~idx? \extract coll?)) \iff idx? \in coll? \\
  coll! = atIndex(coll?, idx?) = \langle  \rangle \iff idx? \not \in coll?
\end{schema}
Given the definition of the $Collection$ and $V$ free types
\begin{zed}
  Collection :== emptyColl ~| ~append \ldata Collection \cross Scalar ~\lor Collection ~\lor KV \cross \nat \rdata \\
  V ::= Scalar ~| ~Collection ~| ~KV
\end{zed}
The collection member $coll?_{idx?} : V$ is implied from $append$ accepting the argument of type $Scalar ~\lor Collection ~\lor KV \equiv V$ which means each Collection member is of type $V$. Given that extraction ($~\_~\extract~\_$) returns a Collection,
\begin{axdef}
  \seq X : Collection
  \where
  \_~\extract~\_ : \power \nat_1 \cross \seq X \fun \seq X
\end{axdef}
in order for $atIndex$ to return the collection member without altering its type,
the first member of $atIdx'$ must be returned, not $atIdx'$ itself.
\begin{axdef}
  atIdx' : Collection \\
  coll!, coll?_{idx?} : V \\
  \where
  atIdx' = (~idx? \extract coll?) \implies \langle coll?_{idx?} \rangle \\
  coll! = head(atIdx') = coll?_{idx?}
\end{axdef}
The $head$ call is made possible by restricting $idx?$ to be a single numeric value.
\begin{argue}
  idx?, idx' : \nat \\
  \t1 \# ~ idx? = 1 @ (~idx? \extract coll?) = \langle coll?_{idx?} \rangle @ \\
  \t2 (head (~idx? \extract coll?)) = coll?_{idx?} & expected return given $idx?$ \\
  \t1 \# ~ idx' \geq 2 @ (~idx' \extract coll?) = \langle coll?_{idx'_{i}}~..~~coll?_{idx'_{j}} \rangle @ \\
  \t2 (head (~idx' \extract coll?)) = coll?_{idx'_{i}} & unexpected return given $idx'$
\end{argue}
Additionally, if the provided $idx? \not \in coll?$ then an empty Collection will be returned
given that $head$ must be passed a non-empty Collection.
\begin{axdef}
  head : \seq_1 X \fun X
  \where
  idx? \not \in coll? \implies (~idx? \extract coll?) = \langle  \rangle ~ \lnot ~\seq_1
\end{axdef}
The properties of $atIndex$ are illustrated in the following examples.
\begin{argue}
  X = \langle x_{0}, x_{1}, x_{2} \rangle \\
  \t1 x_{0} = 0 \\
  \t1 x_{1} = foo \\
  \t1 x_{2} = \langle a, b, c \rangle \\
  atIndex(X, 0) = 0 & $head~(\langle~x_{0}~\rangle)$\\
  atIndex(X, 1) = foo & $head~(\langle~x_{1}~\rangle)$\\
  atIndex(X, 2) = \langle a, b, c \rangle & $head~(\langle~x_{2}~\rangle)$\\
  atIndex(X, 3) = \langle  \rangle & $3 \not \in X \implies x_{3} \not \in X$
\end{argue}
\end{document}
