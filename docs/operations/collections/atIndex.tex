\documentclass[../../main.tex]{subfiles}
\begin{document}
\subsubsection{At Index}
The operation $atIndex$ will return the Value at a
specified numeric index within a Collection or
an empty Collection if there is no value at the specified index.
\begin{schema}{atIndex[Collection, \nat]}
  idx? : \nat\\
  coll? : Collection \\
  atIndex : Collection \cross \nat \fun V
  \where
  \# ~ idx? = 1 \\
  coll! = atIndex(coll?, idx?) = (head ~(~idx? \extract coll?)) \iff idx? \in coll? \\
  coll! = atIndex(coll?, idx?) = \langle  \rangle \iff idx? \not \in coll?
\end{schema}
\begin{argue}
  X = \langle x_{0}, x_{1}, x_{2} \rangle \\
  \t1 x_{0} = 0 \\
  \t1 x_{1} = foo \\
  \t1 x_{2} = \langle a, b, c \rangle \\
  atIndex(X, 0) = 0 \\
  atIndex(X, 1) = foo \\
  atIndex (X, \langle 1, 0 \rangle) = f & foo $\implies \langle f, o, o \rangle$ \\
  atIndex (X, \langle 1, 2 \rangle) = o & foo $\implies \langle f, o, o \rangle$ \\
  atIndex(X, 2) = \langle a, b, c \rangle \\
  atIndex(X, \langle 2, 1 \rangle) = b \\
  atIndex(X, 3) = \langle  \rangle \\
  atIndex(X, \langle 2, 3 \rangle) = \langle  \rangle
\end{argue}
\end{document}
