\documentclass{article}

%%% Package imports
\usepackage{graphicx}
\graphicspath{{../../resources/plots/} {../resources/plots/} {../../resources/figures/} {../resources/figures/}}
\usepackage{placeins}
\usepackage{subfiles}
\usepackage[utf8]{inputenc}
\usepackage[ruled]{algorithm2e}
\usepackage{hyperref}
\hypersetup{
    colorlinks=true,
    linkcolor=blue,
    filecolor=magenta,
    urlcolor=cyan,
}
\usepackage{amsmath}
\usepackage{zed-csp}
\usepackage{breqn}
\usepackage{xcolor}
\usepackage{listings}
\usepackage{pgfplots}
\usepackage{pgfplotstable}
\pgfplotsset{compat=newest}
\usepgfplotslibrary{dateplot}
\usepgfplotslibrary{polar}
\usetikzlibrary{pgfplots.dateplot}
\usetikzlibrary{pgfplots.patchplots}
\usetikzlibrary{patterns}

\usepackage{floatrow}

\usepackage{calc}
%%% \makeatletter\amparswitchfalse\makeatother\
\DeclareMarginSet{hangleft}{\setfloatmargins
{\hskip-\marginparwidth\hskip-\marginparsep}{\hfil}}
\floatsetup[widefigure]{margins=hangleft}
%%% ^ Figure formatting within Appendex A

\lstset{literate = {-}{-}1} % get dashs to show up

\pgfplotsset{compat=1.15}

\SetKw{KwBy}{by}

\usepackage{titlesec}
\newcommand{\sectionbreak}{\clearpage}

\title{Data Analytics and Visualization Environment for xAPI and the Total Learning Architecture: DAVE Learning Analytics Algorithms}
\author{Yet Analytics}

\begin{document}

\begin{titlepage}
  \maketitle
\end{titlepage}

\section*{Introduction}

This report introduces the initial learning analytics algorithms,
\textbf{timeline of learner success},\textbf{which assessment
  questions are the most difficult} and \textbf{rate of completions}.
In doing so, it establishes a set of style
guidelines for the reporting of algorithms and associated visualization templates.
This document will be updated to include additional learning analytics
questions presented elsewhere within this github repo in addition to other learning
analytics algorithms which have yet to be defined. Updates may also
address refinement of these algorithms and this document should be
understood to be an example of algorithm presentation and not the
final state of any defined algorithm.

$\\\\$
The structure of this documents is as follows:
\begin{enumerate}
\item A formal specification for xAPI written in Z
  and referenced within the formal specifications of learning
  analytics algorithms
\item An algorithm definition which will consist of:
  \begin{enumerate}
  \item an introduction for the algorithm
  \item the structure of the ideal input data
  \item how to retrieve input data from an LRS
  \item the statement parameters which the algorithm will utilize
  \item notices regarding data collected during the 2018 pilot test of
    the TLA
  \item a summary of the algorithm
  \item the formal specification of the algorithm
  \item pseudocode representation of the algorithm
  \item JSONSchema for the output of the algorithm
  \item a description of the associated visualization
  \item a prototype of the visualization
  \item a collection of suggestions describing how the algorithm could be
    adapted to improve the quality of the visualization prototype
  \end{enumerate}
\end{enumerate}

\subfile{z/xapi.tex}



\subfile{algorithms/timeline_learner_success.tex}

\subfile{algorithms/most_difficult_assessment_questions.tex}

\subfile{algorithms/rate_of_completions.tex}

\subfile{algorithms/followed_recommendations.tex}

\subfile{appendices/a.tex}


\end{document}
