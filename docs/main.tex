\documentclass{article}

%%% Package imports
\usepackage{graphicx}
\graphicspath{{../../resources/plots/} {../resources/plots/} {../../resources/figures/} {../resources/figures/}}
\usepackage{placeins}
\usepackage{subfiles}
\usepackage[utf8]{inputenc}
\usepackage[ruled]{algorithm2e}
\usepackage{hyperref}
\hypersetup{
    colorlinks=true,
    linkcolor=blue,
    filecolor=magenta,
    urlcolor=cyan,
}
\usepackage{amsmath}
\usepackage{zed-csp}
\usepackage{breqn}
\usepackage{xcolor}
\usepackage{listings}
\usepackage{pgfplots}
\usepackage{pgfplotstable}
\pgfplotsset{compat=newest}
\usepgfplotslibrary{dateplot}
\usepgfplotslibrary{polar}
\usetikzlibrary{pgfplots.dateplot}
\usetikzlibrary{pgfplots.patchplots}
\usetikzlibrary{patterns}

\usepackage{floatrow}

\usepackage{calc}
%%% \makeatletter\amparswitchfalse\makeatother\
\DeclareMarginSet{hangleft}{\setfloatmargins
{\hskip-\marginparwidth\hskip-\marginparsep}{\hfil}}
\floatsetup[widefigure]{margins=hangleft}
%%% ^ Figure formatting within Appendex A

\lstset{literate = {-}{-}1} % get dashs to show up

\pgfplotsset{compat=1.15}

\SetKw{KwBy}{by}

\usepackage{titlesec}
\newcommand{\sectionbreak}{\clearpage}

\title{Data Analytics and Visualization Environment for xAPI and the Total Learning Architecture: DAVE Learning Analytics Algorithms}
\author{Yet Analytics}

\begin{document}

\begin{titlepage}
  \maketitle
\end{titlepage}

\section*{Introduction}

This report introduces the updated definition of learning analytics algorithms in terms of
\textbf{Operations}, \textbf{Primitives} and \textbf{Algorithms} and will feature an updated definition for
each of the previously defined algorithms. This document will be updated to include additional Operations,
Primitives and Algorithms as they are defined by the Author(s) of this report or by members of the Open Source Community.
Updates may also address refinement of existing definitions, thus this document is subject to continious change
but those which are significant will be documented within the DAVE change log. The formal definitions in this document
are optimized for understandability and are not presented as, or intended to be, the most computationaly effecient definition possible.
The formal definitions are intended to serve as referential documentation of methedologies and programatic strategies
for handling the processing of xAPI data.
$\\\\$
The structure of this documents is as follows:
\begin{enumerate}
\item An Introduction to Z notation and its usage in this document
\item A formal specification for xAPI written in Z % FIXME: review + refactor as necessary
\item Terminology: Operations, Primitives and Algorithms
\item What is an Operation
\item What is a Primitive
\item What is an Algorithm
\item Foundational Operations
\item Common Primitives                 % WIP: - see FIXMEs
\item An algorithm definition including % WIP: - update to Z, blocked by common primitives
  \begin{enumerate}
  \item Init
  \item Relevant?
  \item Accept?
  \item Step
  \item Result
  \end{enumerate}
\end{enumerate}

\subfile{z/introduction.tex}
\subfile{z/xapi.tex}
\subfile{algorithms/introduction.tex}

\section{Foundational Operations}
The Operations in this section use the Operations pulled from the Z Reference Manual (section 1,4) within their own definitions.
They are defined as Operations opposed to Primitives because they represent core functionality needed in the context
of processing xAPI data given the definition of an Algorithm above. As such, these Operations are added to the global
dictionary of symbols usable, without a direct reference to the components schema,
within the definition of Operations and Primitives throughout the rest of this document.

\subsection{Collections}
Operations which expect a Collection $X = \langle x_{i}..x_{n}..x_{j} \rangle$
\subfile{operations/collections/array?.tex}
\subfile{operations/collections/append.tex}
\subfile{operations/collections/remove.tex}
\subfile{operations/collections/atIndex.tex}
\subfile{operations/collections/update.tex}

\subsection{Key Value Pairs}
Operations which expect a Map $M = \ldata k_{i}v_{k_{i}}..k_{n}v_{k_{n}}..k_{j}v_{k_{j}} \rdata$

\subfile{operations/kv/map?.tex}
\subfile{operations/kv/associate.tex}
\subfile{operations/kv/dissociate.tex}
\subfile{operations/kv/atKey.tex}

\subsection{Utility}
Operations which are usefull in many Statement processing contexts.
\subfile{operations/util/map.tex}
\subfile{operations/util/isoToUnix.tex}
\subfile{operations/util/timeUnitToNumberOfSeconds.tex}
\subfile{operations/util/rateOf.tex}
% FIXME: Define atJSONPath as Operation (special case, hence utility categorization).
% - Reference another definition of atJsonPath by URL within copy and establish likeness to getIn
% - in the case of JSONPath, the arg $path?$ is a domain specific structured string but that language is defined outside of this document
% - further details are outside the scope of this document
% - update file path, prev version = \subfile{primitives/atJsonPath.tex}

\section{Common Primitives}
% Low Priority:
% TODO: update file name?
% TODO: split into multiple files?
% TODO: clean~\_ primitive - removes all mappings where V \mapsto \emptyset
\subfile{primitives/walk.tex}

% TODO: remove Collections and Key Value Pair sections after completion of above FIXME:
% - also delete the corresponding .tex files

%\subsection{Collections}
%Primitives which expect a Collection $X = \langle x_{i}..x_{n}..x_{j} \rangle$
%\subfile{primitives/collection/appendAt.tex}

%\subsection{Key Value Pairs}
%Primitives which expect a Map $M = \ldata k_{i}v_{k_{i}}..k_{n}v_{k_{n}}..k_{j}v_{k_{j}} \rdata$
%\subfile{primitives/kv/associateAt.tex}

%\subsection{Utility}
%Primitives which are usefull in many Statement processing contexts.
%\subfile{primitives/accumulate.tex}

\section*{Updated Algorithm Definitions}
% WIP - update to Z using definitions above
The following are examples of the new way in which Algorithms were defined. These sections are either in draft form or are a work in progress.
$\\\\$
[THE FOLLOWING IS OUT OF SYNC WITH REST OF DOCUMENT!]
$\\\\\\\\\\\\$
[REST OF THIS PAGE INTENTIONALLY LEFT BLANK]
\subfile{algorithm_definitions/rateOfCompletions.tex}
\subfile{algorithm_definitions/timelineLearnerSuccess.tex}
\subfile{algorithm_definitions/mostDifficultAssessmentQuestions.tex}
\subfile{algorithm_definitions/followedRecommendations.tex}

%\section*{Previous Algorithm Definitions}
%The following are examples of the previous way in which Algorithms were defined.
%\subfile{algorithms/rate_of_completions.tex}
%\subfile{algorithms/timeline_learner_success.tex}
%\subfile{algorithms/most_difficult_assessment_questions.tex}
%\subfile{algorithms/followed_recommendations.tex}

\subfile{appendices/a.tex}


\end{document}
